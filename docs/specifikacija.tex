\chapter{Specifikācija}
Vadības bloka specifikācija:
\begin{itemize}
    \item Darba punkta iestatīšanas procedūra (\textit{ang.} Bias-Up procedure)
    \begin{itemize}
        \item Iestatīt \( I_{\text{D}} \) ierobežojumu uz 200 mA, \( I_{\text{G}} \) ierobežojumu uz 18 A;
        \item Iestatīt \( V_{\text{G}} \) uz − 5.0 V;
        \item Iestatīt \( V_{\text{D}} \) uz 24 V;
        \item Pielāgot \( V_{\text{G}} \) pozitīvāk, līdz \( I_{\text{DQ}} \) = 3.4 A;
        \item Pielietot RF signālu.
    \end{itemize}
    \item Darba punkta atiestatīšana procedūra (\textit{ang.} Bias-Down procedure)
    \begin{itemize}
        \item Samazināt \( V_{\text{G}} \) līdz − 5.0 V. Pārliecināties, ka \( I_{\text{DQ}} \) \~{} 0 mA;
        \item Iestatīt \( V_{\text{D}} \) uz 0 V;
        \item Izslēgt RF signālu;
        \item Izslēgt \( V_{\text{D}} \) barošanu;
        \item Izslēgt \( V_{\text{G}} \) barošanu.
    \end{itemize}
    \item Caur tīkla vadāmus režīmus
    \begin{itemize}
        \item Jaudas pastiprinātāja un elektrobarokļu ieslēgšana;
        \item Jaudas pastiprinātāja un elektrobarokļu izslēgšana;
        \item Parametru monitorēšana:
        \begin{itemize}
            \item Temperatūru; 
            \item Spriegumu;
            \item Strāvu;
            \item Slodzi;
            \item Režīmu;
    \end{itemize}
        \item Kļūdu paziņošānu un iespēju viegli atkļūdot. 
    \end{itemize}
\end{itemize}

