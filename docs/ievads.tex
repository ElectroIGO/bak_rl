\chapter{Ievads}
Ventspils Starptautiskā Radioastronomijas Centra (\textit{VSRC}) rīcībā ir teleskopi RT-16 un RT-32, kas vēsturiski tika izmantoti rietumu pasaules spiegošanā, bet pašlaik ir zinātnes instruments kosmosa izpētē. VSRC zinātniskajā institūtā pašlaik notiek darbs pie radioteleskopu komercializācijas, izmantojot antenas kā bāzes stacijas satelītu un kosmosa izpētes misijām, t.sk uz Mēness. Šim nolūkam tiek veidots S/X diapazona raiduztvērējs RT-16 radioteleskopam, kur daļa no tā ir X diapazona (No 7.25 GHz līdz 7.75 GHz) raidītājs, ko izstrādā darba vadītājs Mārcis Bleideris. Raidītāja vadībai ir jaizstrādā vadības bloka risinājums, kas ietver sistēmas ieslēgšanu, izslēgšanu X joslas 100 W Gallija nitrīda (\textit{GaN}) jaudas pastiprinātāja (\textit{HPA}) modelim (QPM1017) un elektrobarošanas avotiem. Vadības bloks ir nepieciešams, lai pasargātu augstas jaudas pastiprinātāju no pārkaršanas, pārsprieguma, pārstrāvas vai nesaskaņotas slodzes pretestības radītiem riskiem, kas var radīt neatgriezeniskus iekārtas bojājumus. 
Bakalaura darba mērķa sasniegšanai, tika izvirzīti šādi uzdevumi:
\begin{itemize}
    \item Izpētīt esošās jaudas pastiprinātāju vadības sistēmas;
    \item Piemeklēt specifikācijai piemērotu risinājumu;
    \item Izveidot testa stendu;
    \item Izstrādāt funkcionējošu maketu, kuru var vadīt caur ethernet tīklu;
    \item Izstrādāt iespiedpalti, piemērotu VSRC vajadzībām;
    \item Saintegrēt to eksistējošā korpusā.
\end{itemize}
Turpmāk darbs tiek sadalīts vairākās daļas: teorijā, izstrādē un testēšanā. Teorijā tiek padziļinātāk apskatīts idejas koncepts, apskatīties risinājumi un izvēlētā risinājuma teorētiskais pamatojums. Izstrādē tiek detalizēti izsklāstīts iekārtas veidošanas process no izstrādes platei līdz paštaisītam risinājumam ar nepieciešamajām funkcionalitātēm. Testēšanā tiek pārbaudīta iekārtas atbilstība norādītajai specifikācijai.

