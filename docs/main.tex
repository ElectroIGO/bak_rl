\documentclass[12pt,paper=a4]{report}
\usepackage[utf8]{inputenc}
\usepackage{polyglossia}
%Atkāpes pirmajām rindiņām
\usepackage{indentfirst}
%Virsrakstu noformējumam
\usepackage{titlesec}
%Attēlu importēšanai
\usepackage{graphicx}
%Vajadzēs saliktus attēlus
\usepackage{subcaption}
\usepackage{float}
%Nodefinē attālumus no malām
\usepackage[a4paper, lmargin=3.5cm,rmargin=2cm,tmargin=2cm, bmargin=2cm]{geometry}
%rotēšana
\usepackage{rotating, graphicx}
%Apjomīgām tabulām
\usepackage{longtable}
% tabulu šūnu satura centrēšanai
\usepackage{array}
\newcolumntype{P}[1]{>{\centering\arraybackslash}p{#1}}
\newcolumntype{M}[1]{>{\centering\arraybackslash}m{#1}}
%Numerācijai lappuses labajā pusē
\usepackage{fancyhdr}
%Literatūras saraksta veidošanai
\usepackage{natbib}
\usepackage{url}

%Garas formulas
\usepackage{amsmath}

%tabulas
\usepackage{booktabs}
\usepackage{adjustbox}

% rindu sapludināšana tabulā
\usepackage{multirow}% http://ctan.org/pkg/multirow
\usepackage{hhline}% http://ctan.org/pkg/hhline
%laas tabulas
\usepackage{tabularx}
\usepackage{boldline}
%Lai varetu ievietot lapu landscape
\usepackage{lscape}
% pielikumam
\usepackage[titletoc]{appendix}
%Noformējuma parametri---------------------------------------------------------------------------------------------
%Nodefinējam valodas
\setdefaultlanguage{latvian}
\setotherlanguages{english,russian}

%Fonts un atstarpe starp rindām
\setmainfont{Times New Roman}
%Rindiņas pirmā atkāpe
\setlength{\parindent}{1.27cm}
\usepackage{setspace}
\onehalfspacing
%Apakšmape bildēm
\graphicspath{{./pictures/}{./Schematics/}}

%Krāsa dažādiem ieēnojumiem
%\definecolor{shadecolor}{rgb}{0.75, 0.85, 0.38}
%Virsrakstu formatējums
%\newfontfamily\sffamily{Times New Roman}
% Define Times New Roman as the main sans-serif font
\renewcommand{\sfdefault}{ptm}
%\newfontfamily\russianfont{Times New Roman}
%\titleformat{\chapter}{\large\centering\sffamily\bfseries}{\thechapter}{0pc}{}
%{\chaptertitlename\ \thechapter}{20pt}{\Huge}
\titleformat{\chapter}
  {\centering\sffamily\fontsize{16}{16}\bfseries}{\thechapter \space{}}{0pc}{}
\titleformat{\section}{\sffamily\fontsize{14}{14}\bfseries}{\thesection \space{}}{0pc}{}
\titleformat{\subsection}{\sffamily\fontsize{12}{12}\bfseries}{\thesubsection \space{}}{0pc}{}
\titlespacing*{\chapter}{0pt}{0pt}{10pt} 
\titlespacing*{\section}{0pt}{10pt}{5pt} 
\titlespacing*{\subsection}{0pt}{5pt}{5pt} 

%Nodefinējam objektu numerācijas noteikumus
\def\thechapter      {\arabic{chapter}}
\def\thesection      {\ifx\chapter.\undefined{\arabic{section}}\else  {\thechapter.\arabic{section}}\fi}
\def\thesubsection   {\thesection.\arabic{subsection}}
\def\thesubsubsection{\thesubsection.\arabic{subsubsection}}

\usepackage{pdfpages}

\renewcommand{\thefigure}{\arabic{chapter}.\arabic{figure}.}
\renewcommand{\theequation}{\thechapter.\arabic{equation}.}
\renewcommand{\thetable}{\thechapter.\arabic{table}.}
%Attelu un tabulu nosaukumi
%\captionsetup[figure] {labelformat=default,labelsep=space,justification=centerlast}
%\captionsetup[table] {labelformat=simple,labelsep=newline,textfont=bf}
%\renewcommand{\figurename}{att.}

%\renewcommand{\tablename}{tabula}
%\renewcommand{\appendixname}{Pielikums}
%\makeatletter
%\renewcommand{\fnum@figure}{}
%\makeatother
%Teksta nodaļu fiksētie nosaukumi
\addto\captionslatvian{
\renewcommand\bibname{Izmantotās literatūras un avotu saraksts}
\renewcommand{\contentsname}{Saturs}
}

%Numeracija

\pagestyle{fancy}

\fancyhf{}
\renewcommand{\headrulewidth}{0pt} 
\renewcommand{\footrulewidth}{0pt}
\fancyfoot[R]{\thepage}
\fancypagestyle{plain}{%
  \fancyfoot[R]{\thepage}%
}
%Pārnesumiem - ļauj tiasīt lielākas starpas
\hyphenpenalty=5000
%% Atraitņrindiņas un bāreņrindiņas ( widow orphan) vadība
\clubpenalty10000
\widowpenalty10000
% Apakšsvītrām
\usepackage{underscore}
% pārkares paragrāfiem (speciāli saīsinājumiem)
\newenvironment{hangingpar}[1]
  {\begin{list}
          {}
          {\setlength{\itemindent}{-#1}%%'
           \setlength{\leftmargin}{#1}%%'
           \setlength{\itemsep}{0pt}%%'
           \setlength{\parsep}{\parskip}%%'
           \setlength{\topsep}{\parskip}%%'
           }
    \setlength{\parindent}{-#1}%%
    \item[]
  }
  {\end{list}}

%\renewcommand{\baselinestretch}{1.5}
\fontsize{12pt}{1.2}
\linespread{1.25} %1.2 *1.25 = 1.5

\begin{document}

% % Titullapa
\begin{titlepage}
\begin{center}
\textbf{
VENTSPILS AUGSTSKOLA\\
INFORMĀCIJAS TEHNOLOĢIJU FAKULTĀTE}\\
\vspace{1.2cm}
\textbf{BAKALAURA DARBS}\\
\vspace{1.4cm}
{\LARGE \textbf{X joslas raidīšanas sistēmas vadības bloka prototipa izstrāde}}\\
\vspace{1cm}
\begin{tabular}{@{}r@{}l@{}}
\parbox[c]{0.4\textwidth}{Autors:}&
\parbox[t]{0.6\textwidth}{
Ventspils Augstskola\\
Informācijas tehnoloģiju fakultātes\\
profesionālās bakalaura studiju programmas \\ "Elektronikas inženierija"\\
4. kursa students\\
Rodrigo Laurinovičs \\
Matrikulas~Nr. 190050 \vspace{0.7em}\\
\mbox{}\hrulefill\vspace{-0.4em}\\
{\scriptsize(paraksts)}\vspace{1.2cm}} \\
\parbox[c]{0.4\textwidth}{Fakultātes dekāns:}&
\parbox[t]{0.6\textwidth}{
doc. Dr.sc.comp. Vairis Caune \vspace{.7em}\\
\mbox{}\hrulefill\vspace{-0.4em}\\
{\scriptsize(paraksts)}\vspace{1.2cm}} \\
%-------------------------------------------------------------------------
\parbox[c]{0.4\textwidth}{Zinātniskais vadītājs:}&
\parbox[t]{0.6\textwidth}{
Mg. Sc. Ing. Mārcis Bleideris \vspace{.7em}\\
\mbox{}\hrulefill\vspace{-0.4em}\\
{\scriptsize(paraksts)}\vspace{1.2cm}} \\
%----------------------------------------------------------------------------------
\parbox[c]{0.4\textwidth}{Recenzents:} & %\vspace{.7em}\\
\parbox[t]{0.6\textwidth}{
Mg. Sc. Ing. Artūrs Orbidāns  \vspace{.7em}\\
%\mbox{}\hrulefill\vspace{-0.4em}\\

%{\scriptsize(Ieņemamais amats, zinātniskais nosaukums,
%vārds, uzvārds)}\vspace{2em}

\mbox{}\hrulefill\vspace{-0.4em}\\
{\scriptsize(paraksts)}\vspace{2em}} \\
\end{tabular}
\vfill
Ventspils, 2025
\end{center}
\end{titlepage}
\setcounter{page}{2}
%Vieta anotācijām---------------------------------------------------------------------------
\chapter*{ANOTĀCIJA}
\begin{flushleft}
\textbf{Darba nosaukums:} X joslas raidīšanas sistēmas vadības bloka prototipa izstrāde.\\
\textbf{Darba autors:} Rodrigo Laurinovičs\\
\textbf{Darba vadītājs:} Mg. Sc. Ing. Mārcis Bleideris\\
\textbf{Darba apjoms:} \\
\textbf{Atslēgas vārdi:} 
\end{flushleft}


\chapter*{ABSTRACT}
\begin{flushleft}
\textbf{Title:} Prototype development of the control unit for the X band transmission system\\
\textbf{Author:} Rodrigo Laurinovičs\\
\textbf{Supervising tutor:} Mg. Sc. Ing. Mārcis Bleiders\\
\textbf{Scope of work:} 38 pages, 5 tables, 51 figures, 32 bibliographical references, 0 appendices.
\textbf{Keywords:} X-band transmitter bias-up and bias-down, True RMS power detectors, TCP/IP, satellite communications
\\In the bachelor's thesis, a functional prototype for monitoring and biasing a high-power amplifier has been developed. The prototype is network-controllable and applicable for transmitter control in satellite communications in the X-band.\\

The paper describes the RT16 X-band system, the spectrum allocation of the Deep Space Network, existing control systems for power amplifiers, and power detectors. The achievable parameters and functionality required for the efficient use of transmitters in the existing Irbene radio telescope system have been defined.\\

To facilitate prototyping, each subsystem was developed as an independent module using modern, easily accessible electronic components and materials. Computer-aided design tools were used in the development of the module schematics.\\

Measurements of the key electrical parameters were performed on the manufactured subsystems. The prototype was tested in a transmission system, with performance-related parameter measurements conducted, achieved functionality verified, and conclusions drawn.
\end{flushleft}

%Saturs---------------------------------------------------------------------------------------
\tableofcontents
%Darba ievads bez numerācijas -------------------------------------------------------------------
\newcommand{\abbreviation}[3]{%
    \makebox[4em][l]{#1} --  #2 (\textit{#3})\par
}

\chapter*{Saīsinājumi un to klasifikācija}
\begin{hangingpar}{4em}
\abbreviation{SSC}{Zviedru kosmosa korporācija}{Swedish Space Corporation}
\abbreviation{SPI}{Seriālā perifērā saskarne}{Serial Peripheral Interface}
\abbreviation{VSRC}{Ventspils Starptautiskais Radioastronomijas Centrs}{Ventspils International Radio Astronomy Centre}
\abbreviation{RF}{Radiofrekvence}{Radio frequency}
\abbreviation{GaN}{Gallija nitrīds}{Gallium Nitride}
\abbreviation{DSN}{Dziļā kosmosa tīkls}{Deep Space Network}
\abbreviation{SFCG}{Kosmosa frekvenču koordinēšanas grupa}{Space Frequency Coordination Group}
\abbreviation{ITU}{Starptautiskā Telekomunikāciju savienība}{International Telecommunication Union}
\abbreviation{ADC}{Analogciparu pārveidotājs}{Analog-to-Digital Converter}
\abbreviation{DAC}{Ciparanalogu pārveidotājs}{Digital-to-Analog Converter}
\abbreviation{LNA}{Maztrokšnojošs pastiprinātājs}{Low Noise Amplifier}
\abbreviation{HPA}{Augstas jaudas pastiprinātājs}{High Power Amplifier}
\abbreviation{RX}{Uztveršana}{Reception}
\abbreviation{TX}{Raidīšana}{Transmission}
\abbreviation{ICT}{Informācijas un komunikācijas tehnoloģijas}{Information and Communication Technology}
\abbreviation{CCSDS}{Kosmosa datu sistēmu konsultatīvā komiteja }{Consultative Committee for Space Data Systems}
\abbreviation{NTIA}{Valsts telekomunikāciju un informācijas pārvalde}{National Telecommunications and Information Administration}
\abbreviation{RMS}{Vidējā kvadrātiskā vērtība}{Root mean square}
\abbreviation{IC}{Integrālā shēma}{Integral Circuit}


\end{hangingpar}
\chapter{Ievads}
Ventspils Starptautiskā Radioastronomijas Centra (\textit{VSRC}) rīcībā ir teleskopi RT-16 un RT-32, kas vēsturiski tika izmantoti rietumu pasaules spiegošanā, bet pašlaik ir zinātnes instruments kosmosa izpētē. VSRC zinātniskajā institūtā pašlaik notiek darbs pie radioteleskopu komercializācijas, izmantojot antenas kā bāzes stacijas satelītu un kosmosa izpētes misijām, t.sk uz Mēness. Šim nolūkam tiek veidots S/X diapazona raiduztvērējs RT-16 radioteleskopam, kur daļa no tā ir X diapazona (No 7.25 GHz līdz 7.75 GHz) raidītājs, ko izstrādā darba vadītājs Mārcis Bleideris. Raidītāja vadībai ir jaizstrādā vadības bloka risinājums, kas ietver sistēmas ieslēgšanu, izslēgšanu X joslas 100 W Gallija nitrīda (\textit{GaN}) jaudas pastiprinātāja (\textit{HPA}) modelim (QPM1017) un elektrobarošanas avotiem. Vadības bloks ir nepieciešams, lai pasargātu augstas jaudas pastiprinātāju no pārkaršanas, pārsprieguma, pārstrāvas vai nesaskaņotas slodzes pretestības radītiem riskiem, kas var radīt neatgriezeniskus iekārtas bojājumus. 
Bakalaura darba mērķa sasniegšanai, tika izvirzīti šādi uzdevumi:
\begin{itemize}
    \item Izpētīt esošās jaudas pastiprinātāju vadības sistēmas;
    \item Piemeklēt specifikācijai piemērotu risinājumu;
    \item Izveidot testa stendu;
    \item Izstrādāt funkcionējošu maketu, kuru var vadīt caur ethernet tīklu;
    \item Izstrādāt iespiedpalti, piemērotu VSRC vajadzībām;
    \item Saintegrēt to eksistējošā korpusā.
\end{itemize}
Turpmāk darbs tiek sadalīts vairākās daļas: teorijā, izstrādē un testēšanā. Teorijā tiek padziļinātāk apskatīts idejas koncepts, apskatīties risinājumi un izvēlētā risinājuma teorētiskais pamatojums. Izstrādē tiek detalizēti izsklāstīts iekārtas veidošanas process no izstrādes platei līdz paštaisītam risinājumam ar nepieciešamajām funkcionalitātēm. Testēšanā tiek pārbaudīta iekārtas atbilstība norādītajai specifikācijai.


\chapter{Specifikācija}
Vadības bloka specifikācija:
\begin{itemize}
    \item Darba punkta iestatīšanas procedūra (\textit{ang.} Bias-Up procedure)
    \begin{itemize}
        \item Iestatīt \( I_{\text{D}} \) ierobežojumu uz 200 mA, \( I_{\text{G}} \) ierobežojumu uz 18 A;
        \item Iestatīt \( V_{\text{G}} \) uz − 5.0 V;
        \item Iestatīt \( V_{\text{D}} \) uz 24 V;
        \item Pielāgot \( V_{\text{G}} \) pozitīvāk, līdz \( I_{\text{DQ}} \) = 3.4 A;
        \item Pielietot RF signālu.
    \end{itemize}
    \item Darba punkta atiestatīšana procedūra (\textit{ang.} Bias-Down procedure)
    \begin{itemize}
        \item Samazināt \( V_{\text{G}} \) līdz − 5.0 V. Pārliecināties, ka \( I_{\text{DQ}} \) \~{} 0 mA;
        \item Iestatīt \( V_{\text{D}} \) uz 0 V;
        \item Izslēgt RF signālu;
        \item Izslēgt \( V_{\text{D}} \) barošanu;
        \item Izslēgt \( V_{\text{G}} \) barošanu.
    \end{itemize}
    \item Caur tīkla vadāmus režīmus
    \begin{itemize}
        \item Jaudas pastiprinātāja un elektrobarokļu ieslēgšana;
        \item Jaudas pastiprinātāja un elektrobarokļu izslēgšana;
        \item Parametru monitorēšana:
        \begin{itemize}
            \item Temperatūru; 
            \item Spriegumu;
            \item Strāvu;
            \item Slodzi;
            \item Režīmu;
    \end{itemize}
        \item Kļūdu paziņošānu un iespēju viegli atkļūdot. 
    \end{itemize}
\end{itemize}


\chapter{Teorija}

Šajā nodaļā tiek apskatīti jaudas pastiprinātāji, to vadības sistēmas, izstarotās un atstarotās jaudas detektori, kā arī piedāvātie risinājumi vadības sistēmām, jaudas detektoriem un izvēlētie risinājumi pastiprinātājam, vadības sistēmai un jaudas noteikšanai.

\section{"Koncepts"}
\subsection{Augstas jaudas RF pastiprinātāji}
Augstas jaudas RF pastiprinātāji ir būtisks sastāvdaļā bezvadu sistēmās. To uzdevums ir pastiprināt signālus jaudu, ļaujot tos pārraidīt lielākos attālumos. Šie pastiprinātāji tiek izmantoti dažādās nozarēs, tostarp satelītu sakaros.

Tipisks augstas jaudas RF pastiprinātājs sastāv no vairākām daļām, kuras veic dažādas signāla pastiprināšanas un kondicionēšanas funkcijas. 

Galvenās sastāvdaļas ir:
\begin{itemize}
    \item Ieejas tīkla saskaņošana;
    \begin{itemize}
        \item Nodrošina maksimālu enerģijas pārnesi no signāla avota uz pastiprinātāju;
        \item Pielāgo ieejas signāla avota pretestību pastiprinātāja pirmajai pakāpei;
    \end{itemize}
    \item Priekšpastiprinātājs;
    \begin{itemize}
        \item Nodrošina sākotnējo signāla pastiprināšanu ar nelielu pastiprinājumu;
        \item Maztrokšnojoša pakāpe;
    \end{itemize}
    \item Vidējās jaudas pastiprinātājs;
    \begin{itemize}
        \item Veic lielāko daļu signāla pastiprināšanas;
        \item Parasti sastāv no vairākām tranzistoru pakāpēm kaskādes vai paralēlā konfigurācijā;
        \item Var ietvert atgriezeniskās saites shēmas linearitātes un stabilitātes uzlabošanai; 
    \end{itemize}
    \item Gala jaudas pastiprinātājs (izejas pakāpe):
    \begin{itemize}
        \item Nodrošina visaugstāko jaudas līmeni, bieži diapazonā no W līdz kW.;
        \item Izmanto augstas jaudas tranzistorus (piem., MOSFET, GaN HEMT vai LDMOS) vai vakuuma lampas;
        \item Jānodrošina efektīvu siltuma izkliedi, jo rada lielu enerģijas zudumu siltumā.
    \end{itemize}
    \item Izejas saskaņošanas tīkls:
    \begin{itemize}
        \item Saskaņo pastiprinātāja izejas pretestību ar slodzi (piem., antenu vai pārraides līniju);
        \item Nodrošina efektīvu jaudas pārnesi un samazina signāla atstarojumus.
    \end{itemize}
    \item Barošanas avoti un darba punkta iestatīšanas shēmas:
    \begin{itemize}
        \item Nodrošina nepieciešamo līdzstrāvas barošanu aktīvajām komponentēm;
        \item Var ietvert sprieguma stabilizatorus, DC-DC pārveidotājus un darba punkta iestatīšanas tranzistoriem.
    \end{itemize}
     \item Dzesēšanas sistēma:
     \begin{itemize}
        \item Izmanto radiatorus, ventilatorus vai šķidruma dzesēšanas sistēmas termiskās stabilitātes nodrošināšanai.
    \end{itemize}
\end{itemize}

Augstas jaudas RF pastiprinātājus var klasificēt pēc to darbības režīma un efektivitātes:
\begin{itemize}
    \item A klase: Augsta linearitāte, bet zema efektivitāte (~25-30\%). Izmanto pielietojumos, kur nepieciešama minimāla signāla kropļošana;
    \item B/AB klase: Uzlabota efektivitāte (~50-70\%) ar nelielu nelinearitāti, bieži lieto apraides raidītājos;
    \item C klase: Augsta efektivitāte (~75-85\%), piemērota pielietojumiem, kur neliela signāla kropļošana nav kritiska, piemēram, FM raidītājos;
    \item D/E/F klase: pastiprinātāji ar efektivitāti virs 90\%, izmanto specializētās RF pielietojumos.
\end{itemize}

% atsauces:
% Pozar, D. M. (2011). Microwave Engineering (4th ed.). Wiley.
% Krauss, H. L., Bostian, C. W., & Raab, F. H. (1980). Solid State Radio Engineering. Wiley.
% Cripps, S. C. (2006). RF Power Amplifiers for Wireless Communications (2nd ed.). Artech House.

\subsection{Vadības sistēmas RF jaudas pastiprinātājiem}
Lai gan RF jaudas pastiprinātāji ir atbildīgi par signāla pastiprināšanu, vadības sistēma ir tā, kas nodrošina to optimālu darbību, precīzu jaudas regulēšanu un aizsardzību pret iespējamiem darbības traucējumiem.
RF jaudas pastiprinātājiem ir nepieciešama vadības sistēma vairāku iemeslu dēļ:
\begin{itemize}
    \item Jaudas regulēšana: RF pastiprinātājiem jāspēj dinamiski pielāgot savu jaudu atkarībā no ārējiem faktoriem, piemēram, signāla kvalitātes, tālāka pārraides attāluma vai apkārtējās vides apstākļiem;
    \item Aizsardzība pret pārslodzi: RF pastiprinātāji var tikt bojāti pārmērīgas jaudas dēļ. Vadības sistēma var uzraudzīt pastiprinātāja darbību un aktivizēt aizsardzības režīmus, piemēram, automātisku izslēgšanu vai jaudas samazināšanu, lai novērstu bojājumus;
    \item Efektivitāte un enerģijas patēriņš: Ar efektīvu vadības sistēmu iespējams uzlabot pastiprinātāja energoefektivitāti, optimizējot darbības apstākļus, samazinot liekās enerģijas patēriņu un palielinot sistēmas darbības ilgumu;
\end{itemize}

RF jaudas pastiprinātāju vadības sistēmas tiek pielietotas vairākās jomās:

\begin{itemize}
    \item Telekomunikācijas - Vadības sistēma palīdz pielāgot jaudu atbilstoši tīkla apstākļiem, lai nodrošinātu optimālu pārraidi un izvairītos no traucējumiem;
    \item Kosmosa un satelītu tehnoloģijas - Vadības sistēma ļauj uzraudzīt jaudas līmeni un pielāgot to, ņemot vērā vides apstākļus, piemēram, atmosfēras traucējumus;
    \item Radar sistēmas - Vadības sistēma nodrošina, ka radarā tiek izmantota optimāla jauda, lai uzlabotu precizitāti un samazinātu traucējumus;
    \item RFID un bezvadu sensoru tīkli - Vadības sistēma palīdz regulēt jaudu un nodrošina efektīvu enerģijas patēriņu.;
\end{itemize}

Vadības sistēmas RF jaudas pastiprinātājiem tiek implementētas vairākos veidos:
\begin{itemize}
    \item Automātiskā jaudas regulēšana (AGC) - AGC ir bieži izmantota tehnoloģija, kas ļauj automātiski pielāgot pastiprinātāja jaudu, lai uzturētu optimālu signāla līmeni;
    \item Digitālie signālu procesori (DSP) - Vadības sistēmas bieži izmanto DSP tehnoloģijas, lai apstrādātu un analizētu RF signālus reālajā laikā. DSP ļauj veikt precīzu signāla analīzi un uzraudzību, kas nepieciešama, lai regulētu jaudu un nodrošinātu augstu pārraides kvalitāti;
    \item Vairāku sensoru izmantošana - Lai uzraudzītu dažādus parametrus, piemēram, temperatūru, spriegumu un strāvu, tiek izmantoti vairāki sensori, kas darbojas kopā ar vadības sistēmu, lai nodrošinātu pastiprinātāja aizsardzību un efektīvu darbību;
    \item Kompleksi algoritmi un kontrolējošas loģikas sistēmas - Vadības sistēmas var ietvert algoritmus, kas ņem vērā daudzus faktorus, piemēram, apkārtējo traucējumu līmeni, sistēmas pieprasījumu un jaudas pieejamību. Šie algoritmi ļauj pielāgot pastiprinātāja darbību un uzlabot sistēmas veiktspēju;
\end{itemize}

RF jaudas pastiprinātāji ir svarīgi sastāvdaļa modernās komunikāciju sistēmās, un to efektīva darbība ir atkarīga no labi izstrādātām vadības sistēmām. Šādas sistēmas nodrošina jaudas regulēšanu, aizsardzību, enerģijas efektivitāti un augstu signāla kvalitāti, kas ir nepieciešama daudzās nozarēs, sākot no telekomunikācijām līdz kosmosa tehnoloģijām. Ar vienkāršotu.

\subsection{Teorētiska un tehniskā skaidrojums par izstaroto un atstarotā jaudas detektoriem (Prefl un Pfwd detektoriem)}
P\textsubscript{refl} un P\textsubscript{fwd} noteikšanas shēmas tiek izmantotas RF sistēmās, lai uzraudzītu izstaroto un atstaroto jaudu pārvades līnijā. Šie mērījumi ir būtiski, lai nodrošinātu, ka sistēma darbojas efektīvi. P\textsubscript{refl} attiecas uz atstaroto jaudu uz jaudas pastiprinātāju un P\textsubscript{fwd} attiecas uz izstaroto jaudu no jaudas pastiprinātāja.

\subsubsection{Atstarotās P\textsubscript{refl} jaudas detektori}
Atstarotā jauda ir signāla daļa, kas tiek atgriezta avotā pretestības neatbilstību vai atstarojumu dēļ pārraides līnijā. Atstarotās jaudas noteikšana ir svarīga, lai nodrošinātu, ka sistēma nepiedzīvo ievērojamus zaudējumus atstarošanas dēļ, kas var sabojāt komponentes vai izraisīt neefektivitāti.

Atstarotās jaudas detektora tipi:
\begin{itemize}
    \item \textbf{Uz virziena savienotāju balstīta noteikšana (Directional Coupler-based Detection)};
\end{itemize}

Virziena savienotājs ir visizplatītākā metode atstarotās jaudas mērīšanai. Tas ņem paraugus no nelielas atstarotā signāla daļas un novirza to uz detektoru, parasti diodi vai RF barošanas sensoru.

Virziena savienotājs sadala tajā ienākošo jaudu divos ceļos. Signāls, kas pārvietojas virzienā uz priekšu, tiek savienots no viena porta (izstarotā jauda), bet atstarotais signāls tiek savienots no cita porta (atstarotā jauda). Šo jaudu attiecību nosaka savienotāja konstrukcija, un atstarotā jauda tiek noteikta caur saistītu portu, kas atbilst atstarojumam.
Priekšrocības:
\begin{itemize}
    \item Precīzs un vienkāršs;
    \item Nodrošina vienlaicīgus izstarotā un atstarotā jaudas mērījumus;
    \item Bieži un plaši izmanto sakaru sistēmās.
\end{itemize}
Mīnusi:
\begin{itemize}
    \item Frekvences reakcija ir atkarīga no savienotāja konstrukcijas;
    \item Savienotāja zudums (insertion loss) var samazināt mērījumu precizitāti.
\end{itemize}

\begin{itemize}
    \item \textbf{(Bridge-based Detection [Wheatstone Bridge or Impedance Bridge])};
\end{itemize}
Šī metode izmanto tilta slēgumu, lai izmērītu atstaroto jaudu. Šādās ķēdēs atstarošana izraisa sprieguma nelīdzsvarotību, ko var noteikt, bieži vien izmantojot diodi vai citu taisngriešanas elementu.

Līdzsvarota tilta slēgumā izmanto pretestības saskaņošanas principu, lai noteiktu atstarojumus. Ja nav atspulga (perfekta atbilstība), tilts ir līdzsvarots, un signāla atšķirība neveidojas. Kad rodas atstarojumi, pretestības neatbilstība izraisa nelīdzsvarotību.

Priekšrocības:
\begin{itemize}
    \item Augsta jutība pret nelielām atstarotās jaudas variācijām;
    \item Var strādāt plašā frekvenču diapazonā.
\end{itemize}
Mīnusi:
\begin{itemize}
    \item Augstāka sarežgītība salīdzinot ar citām metodēm;
    \item Jutīgi pret dreifēšanu, īpaši ar temperatūras izmaiņām.
\end{itemize}

\begin{itemize}
    \item \textbf{Diožu jaudas detektori (Diode-based Detectors)}.
\end{itemize}
Vienkāršu diodes detektoru var izmantot kopā ar virziena savienotāju vai tiltu, lai tieši izmērītu atstaroto jaudu.

Diodes taisngriezis uztver RF signālu un pārvērš to līdzstrāvas signālā, ko var izmērīt ar atbilstošu skaitītāju. Līdstrāvas spriegums ir proporcionāls atstarotā signāla jaudas līmenim.

Priekšrocības:
\begin{itemize}
    \item Vienkāršā un rentabla (cost-effective);
    \item Nodrošina atstarotās jaudas mērīšanu reālā laikā.
\end{itemize}
Mīnusi:
\begin{itemize}
    \item Precizitāti ierobežo diodes nelinearitāte, īpaši lieljaudas signāliem;
\end{itemize}

\subsubsection{Izstarotās P\textsubscript{fwd} jaudas detektori}
Izstarotā jauda tiek piegādāta no avota uz antenu vai citu patērētāju. Ir nepieciešams monitorēt izstaroto jaudu, lai novērtētu jaudas pastiprinātāja pastiprinājuma koeficientu un vai darbība ir optimāla.

\begin{itemize}
    \item \textbf{Termiskās jaudas noteicējs (Thermal Power Detectors)};
\end{itemize}
Šie detektori mēra signāla radīto siltumu, kad tas iet caur pretestību. Sildīšanas efekts ir proporcionāls piegādātajai jaudai, un termistors vai termopāris var pārvērst šo temperatūras maiņu lasāmā signālā.

Termiskais detektors mēra temperatūras paaugstināšanos pretestībā, ko izraisa RF signāla izkliedētā jauda. Šī metode ir diezgan precīza, jo tā tieši mēra faktisko jaudu, taču tā relatīvi lēna, un tai nepieciešama rūpīga kalibrēšana.

Priekšrocības:
\begin{itemize}
    \item Precīzs un nodrošina absolūtus jaudas mērījumus;
    \item Var izmērīt gan  izstaroto, gan atstaroto jaudu.
\end{itemize}
Mīnusi:
\begin{itemize}
    \item Lēns reakcijas laiks, kas neatļauj izmantot augstās frekvenču aplikācijās.;
\end{itemize}

\begin{itemize}
    \item \textbf{Diožu jaudas detektori (Diode-based Detectors)};
\end{itemize}
Līdzīgi kā atstarotās jaudas noteikšanā, arī izstaroto jaudu var noteikt, taisngriežot signālu, izmantojot diodes. Signāla ceļā tiek ievietots diodes detektors, un, tā kā diode taisno RF signālu, iegūtais līdzstrāvas spriegums ir proporcionāls momentānajai jaudai.

Priekšrocības:
\begin{itemize}
    \item Vienkārši un lēti;
    \item Ātrs reakcijas laiks.
\end{itemize}
Mīnusi:
\begin{itemize}
    \item Precizitāti var ierobežot diodes īpašības, īpaši augstfrekvences signāliem;
    \item Nepieciešama kalibrēšana dažādiem signālu tipiem.
\end{itemize}

\begin{itemize}
    \item \textbf{Pīķa detektori (Peak Detectors)}.
\end{itemize}
Pīķa detektorus var izmantot sistēmās, kurās jāuzrauga tikai maksimālā jauda. Šie detektori uztver maksimālo momentāno jaudu, bieži izmantojot diodes un kondensatora kombināciju, lai saglabātu maksimālo vērtību.

Maksimālais detektors uztver un notur ieejas signāla maksimālo vērtību. Kad signāls sasniedz maksimumu, kondensators uzlādējas līdz noteiktam spriegumam, un līdzstrāvas izeja atspoguļo maksimālo jaudas līmeni.

Priekšrocības:
\begin{itemize}
    \item Ātra reakcija un precīzi pīķa mērījumi;
    \item Noderīga signāliem ar pārejošiem impulsu RF signāliem.
\end{itemize}
Mīnusi:
\begin{itemize}
    \item Nenodrošina vidējo jaudas vērtību;
    \item Var būt mazāk efektīvs nepārtrauktiem signāliem bez būtiskiem pīķiem.
\end{itemize}

Kuras metodes izmanto kādiem pielietojumiem:
\begin{itemize}
    \item Komunikācijas sistēmās - Uz virziena savienotāju balstīta noteikšana (Directional Couplers);
    \item Augstas jaudas sistēmās (raidītājos, jaudas pastiprinātājos) - Termiskais detektors(Thermal Detectors);
    \item Signāla monitorēšanai un atkļūdošanai - Diožu detektors (Diode-base detector);
    \item Mērījumu iekārtām - Tilta detektori (Bridge-based detection circuits).
\end{itemize}

%atsauces
% Gupta, R. S., & Wadhwa, M. (2014). "Microwave Engineering." - This text provides detailed theoretical explanations and practical applications for various types of RF power detectors, including directional couplers and diode detectors.
% Pozar, D. M. (2012). "Microwave Engineering" (4th ed.). - Offers an in-depth explanation of impedance matching and directional coupler theory, relevant to reflected and forward power detection.
% Krauss, J. D. (2011). "RF and Microwave Radiation Safety." - Covers methods and challenges in power measurement, including forward and reflected power detection in RF systems.

\section{"Apskatītie risinājumi"}
\subsection{Vadības integrālās shēmas priekš jaudas pastiprinātāji}
\begin{table}[h]
    \centering
    \renewcommand{\arraystretch}{1.3}
    \setlength{\tabcolsep}{5pt} % Adjust column spacing
    \begin{adjustbox}{max width=\textwidth}
    \begin{tabular}{|p{3cm}|p{3cm}|p{3cm}|p{3cm}|p{4cm}|}
        \hline
        \textbf{Integrālā shēma} & \textbf{Funkcijas} & \textbf{Pielietojums} & \textbf{Spriegums} & \textbf{Atšķiras} \\
        \hline
        Analog Devices AD7293 & Darba punkta iestatīšana un aizsardzība & RF pastiprinātāju vadība, signālu uzraudzība un aizsardzība & 3.3V -- 5.5V & Integrēts ADC un DAC, termālā aizsardzība, sprieguma un strāvas uzraudzība, trauksmes izvadi \\
        \hline
        Texas Instruments ADC12D1800 & 12-Bitu, 1.8 GSPS Datu pārveidotājs & RF sistēmas ātrai signālu nolasei un uzraudzībai & 3.3V & Ātra nolase, zems enerģijas patēriņš un augsta nolases izšķirtspēja \\
        \hline
        Maxim MAX11200 & 24-Bitu, 4-kanālu ADC ar iebūvētu multiplikatoru & Signālu iegūšanai RF pastiprinātāju vadībā & 2.7V -- 5.5V & Augsta precizitāte, zema enerģija, integrēta multiplikatora funkcionalitāte \\
        \hline
        Analog Devices AD5700 & 16-Bitu, 4-kanālu signālu iegūšanas sistēma & RF sistēmu signālu mērīšanai un kontrolei & 3.0V -- 5.5V & Precīzi signālu mērījumi, integrēti trauksmes izvadi, termālā aizsardzība \\
        \hline
        Linear Technology LTC2378-16 & 16-Bitu, 8-kanālu ADC ar iebūvētu multiplikatoru & RF signālu mērīšanai un uzraudzībai & 2.7V -- 5.5V & Augsta precizitāte, zema trokšņa veiktspēja, integrēti trauksmes izvadi \\
        \hline
        Maxim MAX14733 & 16-Bitu, vairāku kanālu ADC ar iebūvētu multiplikatoru & RF signālu uzraudzībai un vadības sistēmām & 3.0V -- 5.5V & Vairāku kanālu signālu iegūšana, zems trokšņu līmenis, trauksmes izvadi \\
        \hline
        Texas Instruments ADS8900B & 16-Bitu, vienkanālu, zema enerģijas ADC & Precīzai signālu uzraudzībai RF pastiprinātāju vadībā & 2.7V -- 5.5V & Zems aizkaves laiks, programmējami trauksmes sliekšņi un reālā laika monitorēšanai \\
        \hline
        Analog Devices AD7779 & 24-Bitu, vairāku kanālu Sigma-Delta ADC ar trauksmes izvadiem & RF sistēmu signālu uzraudzībai un vadībai & 2.7V -- 5.5V & Augsta izšķirtspēja un precizitāte, vairāku kanālu atbalsts, integrēti trauksmes izvadi \\
        \hline
    \end{tabular}
    \end{adjustbox}
    \caption{GaN pastiprinātāju vadības bloku mikroshēmu piemēri un salīdzinājums}
    \label{tab:gan_control_blocks}
\end{table}

No apskatītajiem RF jaudas vadības integrālajām shēmām tika  izvēlēta Analog Devices ražojums "AD7293" salīdzinoši lētās izmaksas dēļ un nodrošina visu nepieciešamo funkcionalitāti pēc VSRC specifikācijām.

\subsection{Izstarotās un atstarotājs jaudas aktīvās RMS jaudas detektors (true RMS power detector)}
Virziena savienotāji izmanto, lai izmērītu gan izstaroto, gan atstaroto jaudu. Virziena savienotājs ir pasīva četru portu ierīce, kas ļauj enerģiju no vienas pārvades līnijas kontrolētā veidā pārnest uz otru pārvades līniju. Tas atdala signālu divās komponentēs: viens ir izstarotā jauda (plūst tajā pašā virzienā kā signāls), bet otrs ir atstarotā jauda (kas plūst atpakaļ uz avotu, ja ir pretestības neatbilstība).

Virziena savienotājs sastāv no četriem portiem:
\begin{itemize}
    \item Ieejas ports - Tas ir ports, kurā signāls tiek ievadīts savienotājā no raidītāja;
    \item Izejas ports - Ieejas signāla izeja, kas tiek nogādāta uz patērētāju, piemēram, antēnu.
    \item Saistīts port (coupled port) - Šī ir pieslēgvieta, kas ir proporcionāls ieejas jaudai. Tas mēra tiešo jaudu līnijā.
    \item Atsaistīts ports (isolated port) - Šis ports saņem atstaroto signālu, kas ir proporcionāla atstarotajam signālam.
\end{itemize}

Darbības princips:
\begin{itemize}
    \item Izejošās jaudas noteikšānai - Ievades signāls nonāk 1. portā, un lielākā daļa no tā iziet no 2. porta. Neliela signāla daļa tiek padota uz 3 portu.
    \item Atstarotās jaudas noteikšānai - Atstarotā jauda ceļo atpakaļ savienotāju, kur daļa no tā tiek padota 4 portā.
\end{itemize}

Virziena savienotāji balstās uz jaudas dalīšanas un savienošanas principu, izmantojot elektromagnētiskos laukus. Šie savienotāji ir paredzēti darbam plašā frekvenču diapazonā, un saites koeficients (coupling factor) nosaka jaudas daudzumu, kas tiek pārnests uz savienoto portu.

\begin{itemize}
    \item \textbf{saites koeficients}
\end{itemize}
Saites koeficients nosaka, cik liela daļa signāla ir "savienota" no galvenās pārraides līnijas uz savienoto līniju. To parasti izsaka dB (decibelos). Piemēram, 20 dB savienotājs nodod 1/100 daļu no ieejas jaudas uz savienoto portu.
Saites koeficientu definē šādi:
\[
C = 10log(\frac{Pinput}{Pcoupled})
\]
, kur
\begin{itemize}
    \item Pinput ir savienotāja ieejas jauda.
    \item Pcoupled ir jauda, kas savienota ar portu 3 (tiešajai jaudai) vai portu 4 (atstarotai jaudai)
\end{itemize}

\begin{itemize}
    \item \textbf{Virziens}
\end{itemize}
Virzība ir vēl viens kritisks parametrs virziena savienotājiem. Tas norāda, cik efektīvi savienotājs atdala izstaroto un atstarotos signālus. Jo augstāks ir virziens.

Virzību parasti izsaka dB un definē kā saistītās jaudas attiecību starp 3 un 4 portu. Ideālā gadījumā savienotājam ar augstu virzību būs niecīgi traucējumi no atstarotā signāla, mērot uz izstaroto jaudu.

\[
D = 10log(\frac{Pcoupledfromport4}{Pcoupledfromport3})
\]

\begin{itemize}
    \item \textbf{Pretestības saskaņošana}
\end{itemize}
Pretestības saskaņošana starp portiem ir būtiska efektīvai signāla pārvadei. Ideālos apstākļos savienotājs ir konstruēts tā, lai ieejas un izejas porti (1. un 2. ports) būtu saskaņoti ar avota un slodzes pretestību (parasti 50 Ω), samazinot atstarojumus savienotājā. Jebkura pretestības neatbilstība palielina atstaroto jaudu, kas var pasliktināt jaudas mērījumu precizitāti.
\begin{itemize}
    \item \textbf{Pārvades līnija}
\end{itemize}
Virziena savienotājs darbojas, izmantojot pārvades līnijas savienojumu, kur divas pārvades līnijas ir novietotas cieši kopā tā, lai elektromagnētiskie lauki no vienas līnijas ietekmētu otru līniju. Sasaistes mehānisma pamatā ir pārvades līniju savstarpējā induktivitāte. Signāli no 1. porta inducē strāvas savienotajā līnijā (3. vai 4. ports), kas ir proporcionāli ieejas jaudai.

\begin{itemize}
    \item \textbf{Jaudas mērīšana}
\end{itemize}

RMS jaudas detektori parasti izmanto virziena savienotājos, lai izmērītu gan uz izstaroto, gan atstaroto jaudu pārvades līnijā. RMS detektors precīzi mēra jaudu, pārveidojot RF signālu līdzstrāvas spriegumā, un pēc tam mērījums tiek apstrādāts, lai noteiktu patieso vidējo jaudu laika gaitā, ņemot vērā gan signāla lielumu, gan fāzi.

Praktiskās sistēmās RMS jaudas detektoru var izmantot, lai izmērītu nesinusoidālus signālus vai signālus ar dažādām amplitūdām.

Virziena savienotāji tiek plaši izmantoti jaudas uzraudzībai un kontrolei RF sistēmās. Daži no pielietojumiem:
\begin{itemize}
    \item Atstarotās jaudas mērīšana.
    \item Izstarotās jaudas mērītāji.
    \item VSWR (sprieguma stāvošo viļņu attiecība) mērījums - Izstaroto un atstarotās jaudas attiecību, kas pazīstama kā VSWR, var aprēķināt, izmantojot mērījumus no virziena savienotāja. Augsts VSWR norāda uz sliktu pretestības atbilstību, kas var izraisīt jaudas zudumus un neefektivitāti;
\end{itemize}

Priekšrocības:
\begin{itemize}
    \item Vienlaicīga izstarotā un atstarotās jaudas mērīšana;
    \item Plašs frekvenču diapazons - Daudzi virziena savienotāji ir paredzēti darbam plašā frekvenču diapazonā;
    \item Uzticamība un precizitāte - nesatur aktīvas komponentes (sastāv no pasīvajām);
    \item Rentablu (cost-effective) - Lēti salīdzinot ar citām metodēm, kur tiek izmantots aktīvi elementi.
\end{itemize}
Mīnusi:
\begin{itemize}
    \item Ievietošanas zudums (insertion loss) - Savienotāja klātbūtne pārvades līnijā rada nelielu zudumu, samazinot kopējo sistēmas efektivitāti.
    \item Pretestības neatbilstība - Ja virziena savienotājs nav pareizi pieskaņots sistēmai, atstarojumi var rasties pašā savienotājā, ietekmējot mērījumu precizitāti;
\end{itemize}

Virziena savienotāji darbojas, pamatojoties uz elektromagnētiskās principiem, un ir ļoti svarīgi tādiem lietojumiem kā jaudas uzraudzība, pretestības saskaņošana un sistēmas diagnostika. Apvienojumā ar True RMS detektoriem tie nodrošina ļoti precīzus jaudas mērījumus signāliem ar dažādām amplitūdām, padarot tos būtiskus sistēmas efektivitātes uzturēšanai un pretestības neatbilstības izraisītu bojājumu novēršanai. Virziena savienotāju priekšrocības, tostarp vienlaicīga mērīšana, augsta virzība un plašs frekvenču diapazons, padara tos par izvēles iespēju daudzās RF pielietojumos, neskatoties uz to ierobežojumiem ievietošanas zudumā un dinamiskajā diapazonā.

\section{"Izvēlētais risinājums"}
\subsection{X-joslas jaudas pastiprinātājs}
\begin{figure}[H]
	\centering
    \includegraphics[width=0.9\textwidth]{bildes/HPA.jpg}\hspace{1cm}
    \caption{HPA augsta līmeņa blokshēma}
\end{figure}
(Informācija no M. Bleidera).
\subsection{Monitoringa sistēmas jaudas pastiprinātājiem}

\subsection{Aktīvā vidējā kvadrātiskā vērtības jaudas noteikana}
\chapter{X joslas raidīšanas sistēmas vadības bloka prototipa izstrāde}

Šajā nodaļā ir izklāstīta šī bakalaura darba ietvaros veiktā X joslas raidīšanas sistēmas vadības bloka prototipa izstrāde. 

\section{Section}

\subsection{Subsection}

\chapter{Testi}
Testi tika sadalīti vairākās daļās, lai pārliecinātos par atsevišķu apakšsistēmu funkcionalitāti. Visi testi izriet no specifikācijas, lai pārbaudītu sistēmu atbilstību. Testi ietver darba punkta iestatīšanu RF jaudas pastiprinātājam, digitālās loģikas barošanas risinājuma novērtējumu, jaudas detektora elektrisko parametru noteikšanu pie 7.2 GHz un tīkla vadību ar TCP protokolu. Testos tiek izmantotas tādas mēriekārtas kā Metex M-3604D digitālais rokas multimetrs \cite{metex_multimeter}, Fluke 87 digitālais multimetrs \cite{fluke_multimeter}, SIGLENT SDS-1104X-U osciloskops \cite{oscil}, Rohde\&Schwarz signāla ģenerators SMP04 \cite{rs_signal_generator}, Rohde\&Schwarz VNA ZVK \cite{rs_vna}, BaseTech BT-305 barošanas avots \cite{test_powersupply} un IGENT SPL 1020 XE 200W DC programmējamo slodzi \cite{programmable_load}.
Testa sadaļās:
\begin{itemize}
    \item Darba punkta iestatīšanas/attiestatīšanas.
    \item 5 un 9 V barošanas risinājuma novērtējums.
    \item Jaudas detektora un koaksiāļo kabeļu novērtējums.
    \item Sistēmas vadība caur tīklu.
\end{itemize}

\section{HPA darba punkta iestatīšana/atiestatīšana}
Darba punkta noteikšanai (skat. 3.1. att.) tika mērīta noteces strāva , 12  un 24 V industriālā barošanas avota līnijas, ieslēgšanas/izslēgšanas slēdža vadības signāls un RF jaudas pastiprinātāja aizvara spriegums. Noteces strāva jaudas pastiprinātājam tiek mērīta ar multimetru, bet pārējie sistēmas parametri - ar osciloskopu.
\begin{figure}[H]
	\centering
    \includegraphics[width=0.8\textwidth]{pictures/test_diagram.png}\hspace{1cm}
    \caption{Darba punkta iestatīšanas/atiestatīšanas testa diagramma}
\end{figure}
Visās oscilogrammās 1. kanāls ir 12 V līnija, 2. kanāls ir 24 V līnija, 3. kanāls ir ieslēgšanas/izslēgšanas vadības signāls un 4. kanāls ir RF jaudas pastiprinātāja aizvara spriegums.
\begin{figure}[H]
	\centering
    \includegraphics[width=0.9\textwidth]{pictures/device_startup.png}\hspace{1cm}
    \caption{Sistēmas ieslēgšana}
\end{figure}
 3.2. att. ieslēdzoties sistēmai, tiek ieslēgts 12 V barošanas avots, kas nodrošina jaudu visiem 9 V un 5 V patērētājiem, tālāk jaudas pastiprinātāja aizvara spriegums tiek iestatīts uz -5 V, lai to aizvērtu. 24 V barošanas avots netiek ieslēgts, jo to dara  mikrokontrolleris ar manuālu vai tīkla komandu, tāpēc arī nevar novērot 24 V vadības spriegumu ieslēgšanas/izslēgšanas slēdzim.
\begin{figure}[H]
	\centering
    \includegraphics[width=0.9\textwidth]{pictures/load_nocap.png}\hspace{1cm}
    \caption{Sistēmas ieslēgšanās oscilogramma ar testa lauktranzistoru}
\end{figure}
\begin{figure}[H]
	\centering
    \includegraphics[width=0.9\textwidth]{pictures/capacitive_load.png}\hspace{1cm}
    \caption{Sistēmas ieslēgšanās oscilogramma ar testa lauktranzistoru un filtra kondensatoru}
\end{figure}
3.3. un 3.4. attēlā var redzēt darba punkta iestatīšanu, kur 3.4 attēlā tiek simulēts pats RF pastiprinātājs, pieliekot klāt kondensatorus filtrs. Kad tiek saņemta starta komanda mikrokontrollerī, tad tiek ieslēgts 24 V barošanas avots, kur momentā tiek nodrošināts aizvara spriegums ieslēgšanas/izslēgšanas p-kanāla lauktranzistoram, lai tas būtu aizvērtā stāvoklī. Tad pēc pāris milisekundēm tiek atvērts lauktranzistors un pēc datu lapas nodrošināta vismaz 100 milisekunžu aizture pēc p-kanāla lauktranzistora aktivizēšanas, lai nostabilizētos pārējie procesi.
\begin{figure}[H]
	\centering
    \includegraphics[width=0.9\textwidth]{pictures/load1load2_f.jpg}\hspace{1cm}
    \caption{Noteces strāvas kreisajam un labajam plecam RF jaudas pastiprinātājam}
\end{figure}
3.5. attēlā var redzēt noteces strāvas abiem RF jaudas pastiprinātāja pleciem. Kreisajā pusē 2.98 A tiek nodrošināti testa tranzistoram bez kapacitatīvas slodzes un labajā pusē 2.95 A ar kapacitatīvu slodzi, kas simulē RF jaudas pastiprinātāju.
\begin{figure}[H]
	\centering
    \includegraphics[width=0.9\textwidth]{pictures/load_off_nocap.png}\hspace{1cm}
    \caption{Sistēmas izslēgšana oscilogramma ar testa lauktanzistoru}
\end{figure}
\begin{figure}[H]
	\centering
    \includegraphics[width=0.9\textwidth]{pictures/cap_load_off.png}\hspace{1cm}
    \caption{Sistēmas ieslēgšanās oscilogramma ar testa lauktanzistoru un filtra kondensatoru}
\end{figure}
3.6. un 3.7. att. var redzēt sistēmas izslēgšanās procesu. Kad tiek saņemta komanda mikrokontrollierī par sistēmas atslēgšanu, tad RF jaudas pastiprinātājam tiek aizvērts, un pēc 100 milisekundēm aizvērts ieslēgšanas / izslēgšanas slēdzis un izslēgts 24 V industriālais barošanas avots.
\begin{figure}[H]
	\centering
    \includegraphics[width=0.8\textwidth]{pictures/test_diagram4.png}\hspace{1cm}
    \caption{Darba punkta iestatīšanas/atiestatīšanas testa diagramma ar differenciālo pāri}
\end{figure}
3.8. att. 1. un 2. kanāls veido diferenciālo pāri, kur tiek mērīts šunta rezistora sprieguma kritums, 3. kanāls mēra vadības signālu ieslēgšanas/izslēgšanas slēdzim un 4. kanāls aizvara spriegumu jaudas pastiprinātājam.
\begin{figure}[H]
	\centering
    \includegraphics[width=0.9\textwidth]{pictures/current.png}\hspace{1cm}
    \caption{Sprieguma krituma noteikšana uz šunta rezistora}
\end{figure}
 Osciloskopam mazākajā mērīšanas diapazonā ir mazāki kanāla trokšņi, tāpēc osciloskopa tausti tika iestatīti uz 10x un osciloskopā nokonfigurēts uz 1x, lai varētu mērīt lielāku signālu ar mazāku sprieguma diapazonu, bet, neskatoties uz to, 1V mērīšanas diapozonam ir 100 mV pk-pk troksnis, tādēļ nebija iespējams izmērīt 22 mV sprieguma kritumu uz šunta rezistora, bet, kad beidzas pārejas process, tad var redzēt, ka troksnis sāk svārstīties ap citu vērtību, bet tas nedod noteiktu vērtību.
\section{Elektrobarošanas plates testi}
Testā tika novērtēts iekārtu jaudas patēriņš un elektrobarošanas risinājuma novērtējums.
\begin{figure}[H]
	\centering
    \includegraphics[width=0.9\textwidth]{pictures/test_diagram2.png}\hspace{1cm}
    \caption{9 un 5 V barošanas testa diagramma}
\end{figure}
3.9. att. redzams testa iekārtu slēgums elektriski principiālajā ķēdē un slodzes pieslēguma vieta.
\begin{table}[H]
\captionsetup{singlelinecheck=off, justification=raggedleft}
\caption{9 V elektrobarošanas}
\centering
\begin{tabular}{|c|c|c|c|c|c|}
\hline
\multicolumn{2}{|c|}{\makecell{Ieejas parametri}} 
& \multicolumn{2}{c|}{\makecell{Izejas parametri}} 
& \multirow{2}{*}{\makecell{Slodze, \si{\ohm}}} 
& \multirow{2}{*}{\makecell{Eff, \%}} \\
\cline{0-3}
\makecell{Spriegums, \si{\volt}} 
& \makecell{Strāva, \si{\ampere}}
& \makecell{Spriegums, \si{\volt}}
& \makecell{Strāva, \si{\ampere}}
&  &\\ 
\hline
12.04\pm0.12 & 0.02\pm0.01 & 8.95\pm0.01 & 0.01\pm0.00 & 14280.00\pm0.00 & 37.16 \\ 
\hline
12.06\pm0.12 & 0.12\pm0.01 & 8.96\pm0.01 & 0.10\pm0.00 & 90.14\pm0.00 & 61.91 \\ 
\hline
12.02\pm0.12 & 0.22\pm0.01 & 8.96\pm0.01 & 0.20\pm0.00 & 44.84\pm0.00 & 67.76 \\ 
\hline
12.03\pm0.12 & 0.32\pm0.02 & 8.98\pm0.01 & 0.30\pm0.00 & 29.90\pm0.00 & 69.98 \\ 
\hline
12.03\pm0.12 & 0.42\pm0.02 & 8.98\pm0.01 & 0.40\pm0.00 & 22.44\pm0.00 & 70.09 \\ 
\hline
12.00\pm0.12 & 0.53\pm0.02 & 8.99\pm0.01 & 0.50\pm0.00 & 17.96\pm0.00 & 70.68 \\ 
\hline
11.99\pm0.12 & 0.63\pm0.02 & 8.99\pm0.01 & 0.60\pm0.00 & 14.98\pm0.00 & 71.40 \\ 
\hline
11.98\pm0.12 & 0.73\pm0.02 & 9.00\pm0.01 & 0.70\pm0.00 & 12.85\pm0.00 & 72.04 \\ 
\hline
11.98\pm0.12 & 0.83\pm0.02 & 9.00\pm0.01 & 0.80\pm0.00 & 11.25\pm0.00 & 72.41 \\ 
\hline
11.97\pm0.12 & 0.93\pm0.03 & 9.00\pm0.01 & 0.90\pm0.00 & 10.01\pm0.00 & 72.76 \\ 
\hline
11.96\pm0.12 & 1.03\pm0.03 & 9.00\pm0.01 & 1.00\pm0.00 & 8.99\pm0.00 & 73.06 \\ 
\hline
\end{tabular}
\end{table}

Izveidotais 9 V pārveidotāja risinājums spēj nodrošināt nepieciešamo jaudu slodzei.

\begin{table}[H]
\centering
\captionsetup{singlelinecheck=off, justification=raggedleft}
\caption{5 V elektrobarošanas}
\begin{tabular}{|c|c|c|c|c|c|}
\hline
\multicolumn{2}{|c|}{\makecell{Ieejas parametri}} 
& \multicolumn{2}{c|}{\makecell{Izejas parametri}} 
& \multirow{2}{*}{\makecell{Slodze, \si{\ohm}}} 
& \multirow{2}{*}{\makecell{Eff, \%}} \\
\cline{0-3}
\makecell{Spriegums, \si{\volt}} 
& \makecell{Strāva, \si{\ampere}}
& \makecell{Spriegums, \si{\volt}}
& \makecell{Strāva, \si{\ampere}}
&  &\\ 
\hline
8.95\pm0.01 & 0.02\pm0.01 & 5.15\pm0.01 & 0.01\pm0.00 & 13020.00\pm0.00 &  28.61 \\ 
\hline
8.96\pm0.01 & 0.11\pm0.01 & 5.15\pm0.01 & 0.10\pm0.00 & 51.22\pm0.00 & 52.02 \\ 
\hline
8.96\pm0.01 & 0.21\pm0.01 & 5.16\pm0.01 & 0.20\pm0.00 & 25.70\pm0.00 & 54.50 \\ 
\hline
8.98\pm0.01 & 0.31\pm0.02 & 5.16\pm0.01 & 0.30\pm0.00 & 17.19\pm0.00 & 55.37 \\ 
\hline
8.98\pm0.01 & 0.41\pm0.02 & 5.16\pm0.01 & 0.40\pm0.00 & 12.89\pm0.00 & 56.10 \\ 
\hline
8.99\pm0.01 & 0.51\pm0.02 & 5.16\pm0.01 & 0.50\pm0.00 & 10.31\pm0.00 & 56.28 \\ 
\hline
8.99\pm0.01 & 0.61\pm0.02 & 5.16\pm0.01 & 0.60\pm0.00 & 8.59\pm0.00 & 56.42 \\ 
\hline
9.00\pm0.01 & 0.71\pm0.02 & 5.15\pm0.01 & 0.70\pm0.00 & 7.35\pm0.00 & 54.87 \\ 
\hline
9.00\pm0.01 & 0.81\pm0.02 & 5.14\pm0.01 & 0.80\pm0.00 & 6.43\pm0.00 & 56.52 \\ 
\hline
9.00\pm0.01 & 0.91\pm0.03 & 5.15\pm0.01 & 0.90\pm0.00 & 5.72\pm0.00 & 56.60 \\ 
\hline
9.00\pm0.01 & 1.01\pm0.03 & 5.15\pm0.01 & 1.00\pm0.00 & 5.14\pm0.00 & 56.66 \\ 
\hline
\end{tabular}
\end{table}

Izveidotais 5 V pārveidotāja risinājums spēj nodrošināt nepieciešamo jaudu slodzei. Slēdzot to kaskādes slēgumā pēc 9 V elektrobarošanas avota, tiek paaugstināta lineārā sprieguma efektivitāte, jo ir mazāk jaudas jāizkliedē uz sevis.

\begin{figure}[H]
\centering
\begin{tikzpicture}
\begin{axis}[
    title={Sprieguma atkarība no slodzes},
    xlabel={Strāva (\si{\ampere})},
    ylabel={Spriegums (\si{\volt})},
    xmin=0, xmax=1,
    ymin=0, ymax=10,
    grid=both,
    grid style={line width=0.1pt, draw=gray!50},
    width=0.8\textwidth,
    axis lines=left,
    legend pos=north west,
]

% Line 1: 9V (Red)
\addplot[
    color=red,
    mark=*,
    line width=1pt,
]
coordinates {
    (0, 8.987)
    (0.1, 8.985)
    (0.2, 8.983)
    (0.3, 8.985)
    (0.4, 8.990)
    (0.5, 8.991)
    (0.6, 8.985)
    (0.7, 8.982)
    (0.8, 8.981)
    (0.9, 8.980)
    (1, 8.979)
};
\addlegendentry{9V}

% Line 2: 5V (Blue)
\addplot[
    color=blue,
    mark=square*,
    dashed,
    line width=1pt,
]
coordinates {
    (0, 5.168)
    (0.1, 5.165)
    (0.2, 5.163)
    (0.3, 5.161)
    (0.4, 5.160)
    (0.5, 5.160)
    (0.6, 5.159)
    (0.7, 5.158)
    (0.8, 5.157)
    (0.9, 5.155)
    (1, 5.153)
};
\addlegendentry{5V}

\end{axis}
\end{tikzpicture}
\caption{Strāvas-sprieguma raksturlīkne 9V un elektrobarošanas līnijām}
\end{figure}
Grafiski atveidoti iegūtie rezultāti. Maksimālais strāvas patēriņš 9 V līnijai ir 310 mA un 5 V līnijai 110 mA.
\section{True RMS jaudas detektora un RF kabeļu tests}
S-parametru noteikšanai tiek veikti no 6.5 līdz 8.5 GHz diapozonā. Tiek veikts koaksiālo kabeļu RFC1 un RFC2 S-parametru (skat. att. 2.1.), kas savieno atzarotāja $P_{fwd}$ un $P_{ref}$ ar jaudas detektoru un jaudas detektora novērtējums. Tad jaudas detektora izejas sprieguma atkarība no ieejas signāla jaudas pie 7.2 GHz.
\begin{figure}[H]
	\centering
    \includegraphics[width=0.7\textwidth]{pictures/cable_coax_fwd.jpg}\hspace{1cm}
    \caption{Izstarotās jaudas koaksiālais kabeļa S-parametri}
\end{figure}
Lielākā daļa no ienākošās jaudas koaksiālajā kabelī netiek atstarota ($S_{11}$ un $S_{22}$) atpakaļ -24 dB, kas ir aptuveni 0.4\% no jaudas, bet, neskatoties uz to, pašā vadā ir -1.8 dB zudums ($S_{21}$ un $S_{12}$), kas ir aptuveni 33\% no ienākošās jaudas.
\begin{figure}[H]
	\centering
    \includegraphics[width=0.7\textwidth]{pictures/cable_coax_ref.jpg}\hspace{1cm}
    \caption{Atstarotās jaudas koaksiālais kabelā S-parametri}
\end{figure}
Lielākā daļa no ienākošās jaudas koaksiālajā kabelī netiek atstarota ($S_{11}$ un $S_{22}$) atpakaļ -20 dB, kas ir aptuveni 1\% no jaudas, bet, neskatoties uz to, pašā kabelī ir -1.9 dB zudums ($S_{21}$ un $S_{12}$), kas ir aptuveni 36\% no ienākošās jaudas.
\begin{figure}[H]
	\centering
    \includegraphics[width=0.7\textwidth]{pictures/tests_diagram3.png}\hspace{1cm}
    \caption{Jaudas detektora testu diagramma}
\end{figure}
3.14. att. redzams testa iekārtu slēgums elektriski principiālajā shēmā.
\begin{figure}[H]
	\centering
    \includegraphics[width=0.7\textwidth]{pictures/vna_measurement_powerdet.jpeg}\hspace{1cm}
    \caption{S-parametri jaudas detektoram}
\end{figure}
Izstarotā un atstarotā jaudas kanāli atstaro aptuveni pusi no ienākošā signāla jaudas. Savstarpējā portu izolācija ir laba, kas tiek panākta ar metalizētiem urbumiem.
\begin{figure}[H]
\centering
\begin{tikzpicture}
\begin{axis}[
    title={Sprieguma līmenis atkarībā no ieejas jaudas},
    xlabel={Jauda (dBm) },
    ylabel={Spriegums (\si{\volt})},
    xmin=-60, xmax=20,
    ymin=0, ymax=2,
    grid=both,
    grid style={line width=0.1pt, draw=gray!50},
    width=0.8\textwidth,
    axis lines=left,
    legend pos=north west,
]

% Line 1: FWD power (Red)
\addplot[
    color=red,
    mark=*,
    line width=1pt,
]
coordinates {
    (-50, 0.104)
    (-49, 0.215)
    (-48, 0.250)
    (-47, 0.279)
    (-46, 0.303)
    (-45, 0.328)
    (-44, 0.350)
    (-43, 0.375)
    (-42, 0.397)
    (-39, 0.466)
    (-36, 0.559)
    (-33, 0.641)
    (-30, 0.754)
    (-27, 0.859)
    (-24, 0.952)
    (-21, 1.048)
    (-18, 1.143)
    (-15, 1.224)
    (-12, 1.347)
    (-9, 1.463)
    (-6, 1.583)
    (-3, 1.672)
    (0, 1.757)
    (3, 1.818)
    (6, 1.863)
    (9, 1.879)
    (10, 1.867)
};
\addlegendentry{Izstarotā jauda}

% Line 2: REF power (Blue)
\addplot[
    color=blue,
    mark=square*,
    dashed,
    line width=1pt,
]
coordinates {
    (-55, 0.355)
    (-54, 0.358)
    (-53, 0.366)
    (-52, 0.368)
    (-51, 0.373)
    (-50, 0.377)
    (-49, 0.382)
    (-48, 0.388)
    (-47, 0.394)
    (-46, 0.403)
    (-45, 0.412)
    (-44, 0.420)
    (-43, 0.431)
    (-42, 0.442)
    (-41, 0.454)
    (-40, 0.466)
    (-39, 0.484)
    (-36, 0.554)
    (-33, 0.636)
    (-30, 0.733)
    (-27, 0.822)
    (-24, 0.914)
    (-21, 1.000)
    (-18, 1.098)
    (-15, 1.175)
    (-12, 1.284)
    (-9, 1.394)
    (-6, 1.503)
    (-3, 1.613)
    (0, 1.686)
    (3, 1.768)
    (6, 1.831)
    (9, 1.920)
    (10, 1.855)
};
\addlegendentry{Atstarotā jauda}

\end{axis}
\end{tikzpicture}
\caption{Izejas sprieguma ietekme uz ieejas jaudu}
\end{figure}
Iegūstot izejas sprieguma līmeņu atkarības no ieejas jaudas, tika ņemti vērā signāla ģeneratora un vada zudumi. Līkne, kas tika iegūta abos kanālos, ir ļoti līdzīga ražotāja dotajai specifikācijai. Jutības diapazons ir no -50 līdz 9 dBm, kur 0 dBm atbilst 100 W, tad mēs varam izmērīt no 1 mW līdz 794,3 W.
\section{Tīkla testi}
Šajā testā tiek pārbaudīta vadības sistēma no operatora datora līdz mikrokontrolieram ar skripta palīdzību, kas rakstīts Python priekš Windows OS. Tests tika veikts izolētā tīklā, kur atradās tikai divas ierīces: operātora dators un mikrokontrolieris ar visu X-joslas sistēmu.
\begin{figure}[H]
	\centering
    \includegraphics[width=0.9\textwidth]{pictures/script_start.png}\hspace{1cm}
    \caption{Klienta konfigurēšana lokālajā tīklā}
\end{figure}
Tests sākās ar skripta uzsākšanu. Terminālī redzams logo, palīgrinda un to, ka veiksmīgi ir pieslēdzies pie TCP servera, kurš uzreiz pēc pieslēgšanās atsūta savu nosaukumu un versiju. Tad lietotājam tiek sniegta informācija par komandu, kura uzsāk monitorēšanu un darba punkta iestatīšanu. Pēc tā tiek aicināts ievadīt komandu.
\begin{figure}[H]
	\centering
    \includegraphics[width=0.9\textwidth]{pictures/script_monchange.png}\hspace{1cm}
    \caption{Monitorēšanas cikla aizkaves maiņa}
\end{figure}
Redzams monitorēšanas intervāla iestatīšanu no noklusētās vērtības 30 sec uz 5 sekundēm.
\begin{figure}[H]
	\centering
    \includegraphics[width=0.9\textwidth]{pictures/script_system_start .png}\hspace{1cm}
    \caption{Darba punkta iestatīšana un telemetrijas virkne}
\end{figure}
Tad tika uzsākta sistēmas darbība ar komandu "start", kur var redzēt, ka tika nosūtīta komanda pon=1 un saņemta atbilde no mikrokontroliera pon=1, kas liecina par to, ka komanda tika saņemta un veiksmīgi apstrādāta. Pēc kā seko monitorēšanas telemetrijas ik pēc 5 sekundēm.
\begin{figure}[H]
	\centering
    \includegraphics[width=0.9\textwidth]{pictures/script_setcurr.png}\hspace{1cm}
    \caption{Darba punkta noteces strāvas iestatīšana}
\end{figure}
\begin{figure}[H]
Tad tiek iestatīts cita darba punkta strāva no 3 A uz 2.45 A, kur mikrokontrolieris arī deva atbildi, ka tā tika veiksmīgi uzstādīta.
	\centering
    \includegraphics[width=0.9\textwidth]{pictures/script_setcurr_tel.png}\hspace{1cm}
    \caption{Strāvas maiņas telemetrijas virkne}
\end{figure}
Tad var redzēt, ka telemetrijas virknē ir aizvara spriegums cits un noteces strāva 2.426 A.
\begin{figure}[H]
	\centering
    \includegraphics[width=0.9\textwidth]{pictures/script_system_stop.png}\hspace{1cm}
    \caption{Darba punkta atiestatīšana}
\end{figure}
Aktīvu sistēmu var deaktivizēt ar komandas nosūtīšanu "stop", kas uzreiz uz mikrokontrolieri nosūta komandu pon=0 un izbeidz monitorēšanas pavedienu.
\begin{figure}[H]
	\centering
    \includegraphics[width=0.9\textwidth]{pictures/script_exit.png}\hspace{1cm}
    \caption{Skripta izslēgšana}
\end{figure}
Ar exit vai ctrl+c var izbeigt skripta darbību, kur tiek aizsūtīta komanda pon=0, lai izslēgtu sistēmu, ja tā bija aktīva, un atvienojas no servera, tad izbeidz visu aktīvo pavedienu darbību.


\chapter{Secinājumi un priekšlikumi}
Sistēmas prototips ir novests līdz stadijai, kad var tikt uzstādīts RT-16 radioteleskopā X-joslas raidīšanas sistēmā. 
Modulārās sistēmas izvēle ar vairākām apakšsistēmām, kuras ir savstarpēji atsaistītas, lai nodrošinātu vienkāršu apakšsistēmu nomaiņu gadījumos, kad sistēma ir bojāta. Tas ātri pārvērtās par grūti pārvaldāmu sistēmu. Tāpēc visas apakšsistēmas nepieciešams integrēt vienā iespiedplatē, samazinot savstarpējo savienojumu skaitu ar vadiem un padarot visu sistēmu vieglāk pārvaldāmu, bet neizdevās laika trūkuma dēļ realizēt.\\
Darba punkta iestatīšana un atiestatīšana atbilst norādītajai tehniskajai specifikācijai. HPA monitorēšanai tiek veiktas sprieguma, temperatūras un caurplūstošas strāvas pārbaudes. Tiek sasniegta 3 A noteces strāva 31.75 ms, lai uzlabotu pārejas procesu HPA. Raidīšanas laikā tiek atslēgta noslēgtā cilpa, kas nodrošina darba punkta iestatīšanas strāvu, lai HPA varētu nodrošināt lielāku jaudu, jo vadības ar atgriezenisko saiti gadījumā, palielinoties noteces strāvai, raidīšanas brīdī darba punkta iestatīšanas integrālā shēma to kompensē, mainot aizvara spriegumu.\\
RMS jaudas detektoram neizdevās nodrošināt -5 dB atstarojuma koeficientu kā norādīts datu lapā, izdevās panākt datu lapā minētu 60 dB mērīšanas diapozonu un noteikt attiecīgo jaudu no 1 mW līdz 100 W. To ir iespējams kompensēt ar ieejas filtra kondensatora vērtības maiņu, ko laika trūkuma dēļ netika izdarīts.\\
Lai palielinātu sistēmas efektivitāti, jāizvieto lineārā sprieguma stabilizatori ar ekvivalentiem impulsa tipa stabilizatoriem.
Tika arī nodrošināta tīkla vadīšana caur TCP protokolu un Python skripts sistēmas aktivizēšanai un konstantas telemetrijas virknes saņemšanai un atveidošanai terminālī cilvēkam saprotamākā reprezentācijā.\\
Jaudas detektorus nepieciešams izstrādāt uz atsevišķām iespiedplatēm, lai nodrošinātu labāku savstarpējo kanālu izolāciju, lai gan tā tagad arī ir ļoti laba, bet ja nebūtu uz vienas iespiedplates, tad būtu -70 dB.\\
Nepieciešams izveidot komandas, kur var nepieciešamības gadījumā manuāli aktivizēt/deaktivizēt noteiktas sistēmas daļas darba punkta un monitorēšanas IC, lai varētu sistēmu padarīt universālu, nevis tikai šim noteiktajam jaudas pastiprinātājam.\\


%\includepdf[pages={1}]{literSaraksts.pdf}
%\includepdf[pages={2}]{literSaraksts.pdf}
%\includepdf[pages={3}]{literSaraksts.pdf}

%\bibliography{litSaraksts}{}
%\bibliographystyle{ieeetr}
%\addcontentsline{toc}{chapter}{Izmantotās literatūras un avotu saraksts}

%\input{pielikumi.tex}

%\input{Annex.tex}
%\input{Galvojums.pdf}

\end{document}

