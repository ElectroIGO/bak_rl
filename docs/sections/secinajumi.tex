\chapter{Secinājumi un priekšlikumi}
Sistēmas prototips ir novests līdz stadijai, kad var tikt uzstādīts RT-16 radioteleskopā X-joslas raidīšanas sistēmā. 
Modulārās sistēmas izvēle ar vairākām apakšsistēmām, kuras ir savstarpēji atsaistītas, lai nodrošinātu vienkāršu apakšsistēmu nomaiņu gadījumos, kad sistēma ir bojāta. Tas ātri pārvērtās par grūti pārvaldāmu sistēmu. Tāpēc visas apakšsistēmas nepieciešams integrēt vienā iespiedplatē, samazinot savstarpējo savienojumu skaitu ar vadiem un padarot visu sistēmu vieglāk pārvaldāmu, bet neizdevās laika trūkuma dēļ realizēt.\\
Darba punkta iestatīšana un atiestatīšana atbilst norādītajai tehniskajai specifikācijai. HPA monitorēšanai tiek veiktas sprieguma, temperatūras un caurplūstošas strāvas pārbaudes. Tiek sasniegta 3 A noteces strāva 31.75 ms, lai uzlabotu pārejas procesu HPA. Raidīšanas laikā tiek atslēgta noslēgtā cilpa, kas nodrošina darba punkta iestatīšanas strāvu, lai HPA varētu nodrošināt lielāku jaudu, jo vadības ar atgriezenisko saiti gadījumā, palielinoties noteces strāvai, raidīšanas brīdī darba punkta iestatīšanas integrālā shēma to kompensē, mainot aizvara spriegumu.\\
RMS jaudas detektoram neizdevās nodrošināt -5 dB atstarojuma koeficientu kā norādīts datu lapā, izdevās panākt datu lapā minētu 60 dB mērīšanas diapozonu un noteikt attiecīgo jaudu no 1 mW līdz 100 W. To ir iespējams kompensēt ar ieejas filtra kondensatora vērtības maiņu, ko laika trūkuma dēļ netika izdarīts.\\
Lai palielinātu sistēmas efektivitāti, jāizvieto lineārā sprieguma stabilizatori ar ekvivalentiem impulsa tipa stabilizatoriem.
Tika arī nodrošināta tīkla vadīšana caur TCP protokolu un Python skripts sistēmas aktivizēšanai un konstantas telemetrijas virknes saņemšanai un atveidošanai terminālī cilvēkam saprotamākā reprezentācijā.\\
Jaudas detektorus nepieciešams izstrādāt uz atsevišķām iespiedplatēm, lai nodrošinātu labāku savstarpējo kanālu izolāciju, lai gan tā tagad arī ir ļoti laba, bet ja nebūtu uz vienas iespiedplates, tad būtu -70 dB.\\
Nepieciešams izveidot komandas, kur var nepieciešamības gadījumā manuāli aktivizēt/deaktivizēt noteiktas sistēmas daļas darba punkta un monitorēšanas IC, lai varētu sistēmu padarīt universālu, nevis tikai šim noteiktajam jaudas pastiprinātājam.\\
