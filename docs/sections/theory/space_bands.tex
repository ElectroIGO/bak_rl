\section{Dziļā kosmosa tīkla spektra sadalījums}
DSN \cite{nasajplfreq} ir izstrādājis frekvenču sadalījuma kanālu plānus, lai ērti un pārskatāmi var apskatīt dziļās kosmosa misijas spektrālo sadalījumu (B kategorija, lielāka par 2 miljoniem km no Zemes) S, X un Ka joslām saskaņā ar SFCG ieteikumiem. Plāni pieļauj vienlaicīgu fāzes saskaņotu augšupsaiti (Zeme-kosmoss, \textit{Uplink}) un lejupsaites (kosmoss-Zeme,\textit{Downlink}) pārraides, kur augšupsaite un lejupsaite ir vienā vai dažādās joslās.\\
ITU piešķir un regulē frekvenču spektru gan komerciālām, gan valstiskām vajadzībām. ITU galvenais mērķis ir koordinēt telekomunikāciju tīklu un pakalpojumu darbību visā pasaulē, pārvaldīt radiofrekvenču spektra un satelīta orbītu sadali, kā arī veicināt piekļuvi ICT, lai atbalstītu ilgtspējīgu attīstību.\\
CCSDS ir starptautiska kosmosa aģentūru organizācija, kas izstrādā un koordinē standartus pārraidei un datu sistēmām, lai atbalstītu kosmosa pētniecību. NTIA, ASV Tirdzniecības departamenta aģentūra, ir izpildvaras galvenā iestāde vietējiem un starptautiskiem telekomunikāciju un informācijas tehnoloģiju jautājumiem. NTIA novērtējumu pamatā ir tehniskie un regulatīvie kritēriji efektīvas un koordinētas frekvenču spektra izmantošanas kosmosa misijām. Apakšējā tabula apskata esošo un apstiprināto frekvenču sadalījumu joslās, kas arī iedalāt pēc attāluma no zemes.

\begin{table}[H]
\centering
\captionsetup{singlelinecheck=off, justification=raggedleft}
\caption{Frekvenču sadalījums joslā}
\begin{tabular}{|c|c|c|c|c|}
\hline
\multirow{2}{*}{Josla} 
& \multicolumn{2}{c|}{\makecell{Dziļā kosmosa josla \\ (2 milijoniem km no zemes)}} 
& \multicolumn{2}{c|}{\makecell{Dziļā kosmosa josla \\ (pēc 2 milijoni km no zemes)}} \\ 
\cline{2-5}
& \makecell{Augšupsaite \\ "zeme-kosmoss"} % Uplink
& \makecell{Lejupsaite \\ "kosmoss-zeme"} % Downlink
& \makecell{Augšupsaite \\ "zeme-kosmoss"} % Uplink
& \makecell{Lejupsaite \\ "kosmoss-zeme"} % Downlink
\\ 
\hline
S josla & 2110-2120 & 2290-2300 & 2025-2110 & 2200-2290 \\ 
\hline
X josla & 7145-7190 & 8400-8450 & 7190-7235 & 8450-8500 \\ 
\hline
K josla & N/A & N/A & 22550-23150 & 25500-27000 \\ 
\hline
Ka josla & 34200-34700 & 31800-32300 & N/A & N/A \\
\hline
\end{tabular}
\end{table}
