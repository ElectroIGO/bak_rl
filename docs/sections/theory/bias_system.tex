\section{Vadības integrālās shēmas RF jaudas pastiprinātājiem}
Zemes bāzes stacijas jaudas patiprinātājiem \cite{DisIntCPA} veiktspējas uzraudzība un kontrole ļauj maksimāli palielināt izejas jaudu, vienlaikus panākot optimālu linearitāti un efektivitāti. Bāzes stacijās tiek izmantoti daudzkanālu DAC, ADC, temperatūras sensori un strāvas mērīšanas integrālās shēmas, kā arī vienas mikroshēmas integrētie risinājumi, lai uzraudzītu un kontrolētu dažādus analogos signālus. Diskrēti sensori un datu pārveidotāji nodrošina maksimālu veiktspēju un konfigurācijas elastību, savukārt integrētie risinājumi piedāvā zemākas izmaksas, mazāku izmēru un lielāku uzticamību. Darba ietvaros tika apskatīti integrālie risinājumi.

\subsection{Apskatītie risinājumi}

Analog Devices AD7293\cite{ad7293} ir jaudas pastiprinātāja kontrolieris, kas ietver funkcionalitāti strāvas, sprieguma un temperatūras vispārīgai uzraudzībai un vadībai ar SPI saskarni. Ir 4 strāvas mērīšanas sensori ar mainīgu pastiprinājumu, atļautais sprieguma diapazons no 4 līdz 60 V un +/- 200 mV sprieguma atšķirību starp pozitīvo un negatīvo izvadu, četri 12 bitu ADC ar mainīgiem references spriegumiem, astoņi 12 bitu DAC ar iestatāmu konversijas laiku līdz 1.3 μs, četri bipolārie un četri unipolārie ar konfigurējamiem izvades diapazoniem, integrēts temperatūras sensors un četri ārējo temperatūras sensora pieslēgvietas, astoņi vispārīgās nozīmes ievadi/izvadi un ieslēgšanas/atslēgšanas shēmas vadības izvads. Ir vadība ar atgriezenisko saiti funkcionalitāte ar iestatāmu konversijas laiku. Iekārtā ir ierobežojumu reģistri ar brīdinājuma funkcionalitāti.

Texas Instruments AMC7834\cite{amc7834} ir jaudas pastiprinātāja kontrolieris, kas ietver funkcionalitāti strāvas, sprieguma un temperatūras vispārīgai uzraudzībai un vadībai ar SPI saskarni. Ir 2 strāvas mērīšanas sensoru no 4 līdz 60 V potenciālu un +/- 200 mV sprieguma atšķirība starp pozitīvo un negatīvo izvadu, 4 bipolārie un 4 unipolārie DAC ar mainīgiem sprieguma diapozoniem, iekšējais temperatūras sensors un 2 ārējās sensora pieslēgvietas, 4 vispārīgās nozīmes ievadi/izvadi. 

Analog Devices AD7294\cite{ad7294} ir jaudas pastiprinātāja kontrolieris, kas ietver funkcionalitāti strāvas, sprieguma un temperatūras vispārīgai uzraudzībai un vadībai ar I2C saskarni. Ir 2 strāvas mērīšanas sensoru no 5V līdz 59.4 V potenciāla un +/- 200 mV sprieguma atšķirība starp pozitīvo un negatīvo izvadu, 4 unipolāri DAC, 4 ADC ieejas un 2 ārējie temperatūras sensoru pieslēgvietas, integrēts temperatūras sensors un 1 brīdinājuma izvads. 3 μs konversijas laiks ADC. AD7294 ietver arī ierobežojumu reģistrus trauksmes funkcijām.

Texas Instruments AMC7908\cite{amc7908} ir jaudas pastiprinātāja kontrolieris, kas ietver funkcionalitāti strāvas, sprieguma un temperatūras vispārīgai uzraudzībai un vadībai ar I2C vai SPI saskarni. Ir 2 8 bitu ADC, astoņi bipolāri 12 bitu DAC, 2 strāvas mērīšanas sensori, temperatūras sensors, divas ārējo temperatūras sensoru pieslēgvietas, četri konfigurējami vispārīga mērķa ievades/izvades porti.

No apskatītajiem RF jaudas vadības integrālajām shēmām tika  izvēlēta Analog Devices AD7293 daudzo funkciju dēļ. Vairāk vispārīgās nozīmes izvadi, kur var ieprogrammēt karodziņus (alert). Ar plašām konfigurācijas iespējām un ieslēgšanas/izslēgšanas shēmu. Var pievienot četrus ārējos temperatūras sensorus. Izceļas ar detalizētu datu lapu. Pieejama izstrādes plate ar vadības kontrolieri un lietojumprogrammatūru.