\section{Jauadas detektori}
Lieljaudas bezvadu sistēmas ir nepieciešams uzraudzīt un kontrolēt gan pārraidīto, gan saņemto RF jaudu \cite{rfpowerdetector}, kas tiek panākts ar RF detektoriem \cite{rfpowerdetector2}, kas tiek visbiežak izmantots bezvadu sistēmās, lai izstarotā jauda atbilstu signāla stipruma normatīvajām prasībām. Lai mērītu RF jaudu, tiek izmantotas trīs dažādas metodes: pīka detektors, logaritmiskais pastiprinātājs un RMS detektors. Katrai metodei tiek pielietota pēc sistēmas vajadzības. Lieljaudas pārraides sistēmās jauda tiek mērīta netiešā veidā, lai pēc iespējas mazāk ietekmētu sistēmu. RF jaudas mērīšanai neizmanto digitālas metodes, jo tās ir lēnākas nekā analogās metodes un ar augstāku jaudas patēriņu, lai sasniegtu līdzvērtīgu rezultātu. Logaritmiskais pastiprinātājs balstās uz diodes raksturīgo logaritmisko reakciju, kas rada izejas spriegumu, kas proporcionāls ieejas signāla jaudai, tāpēc tā izeja ir lineāra attiecībā pret ieeju decibelos. To plaši izmanto, īpaši gadījumos, ja signālam ir plašs dinamiskais diapazons. Tāpēc logaritmiskais pastiprinātājs tiek izmantots sistēmās ar impulsu signāliem, kā radariem. RMS detektors ģenerē spriegumu, kas proporcionāls signāla jaudas RMS vērtībai. RMS detektors ir piemērots signāliem ar augstu signāla amplitūdas pīķa-vidējās vērtības attiecību (crest-factor). Raidītāji ar amplitūdas pīķa-vidējās vērtības attiecību (10-15 dB) kļūst arvien izplatītāki, jo tiek plašāk izmantotas modernas modulācijas metodes, piemēram, augstākas kārtas QAM, CDMA sistēmās vai OFDM. 

Tā kā satelītkomunikācijām tiek izmantota augsta amplitūdas pīķa-vidējās vērtības attiecība modulācija, tiek izvēlēts RMS detektors kā jaudas mērīšanas risinājums.

\subsection{Apskatītie risinājumi}

LTC5582\cite{ltc5582} ir RMS jaudas detektors, kas darbojas frekvenču diapazonā no 40 MHz līdz 10 GHz un ar plašu mērīšanas diapazonu 62 dB. Tas darbojas frekvenču diapazonā no 10 MHz līdz 10 GHz un var apstrādāt ieejas signālus no –60 dBm līdz +2 dBm ar dažādiem amplitūdas pīķa-vidējās vērtības attiecībām.

ADL5906\cite{adl5906} ir RMS jaudas detektors, kas darbojas frekvenču diapazonā no 10 MHz līdz 10 GHz un ar plašu mērīšanas diapazonu 67 dB. Tas darbojas frekvenču diapazonā no 10 MHz līdz 10 GHz un var apstrādāt ieejas signālus no –65 dBm līdz +8 dBm ar dažādiem amplitūdas pīķa-vidējās vērtības attiecībām un joslas platumiem. Piedāvāta temperatūras stabilitāte plašā temperatūras diapazonā (no –55°C līdz +125°C). Nepieciešams 5 V barošanas avots un darba strāva ir 68 mA pie 25°C temperatūrās.

LTC5587\cite{ltc5587id} ir  RMS jaudas detektors ar integrētu 12 bitu seriālo ADC, kas darbojas frekvenču diapazonā no 10 MHz līdz 6 GHz. Mērīšanas diapazons ir no –34 dBm līdz 6 dBm. Detektora seriālā digitālā izeja nodrošina 12 bitu skaitlisko vērtību, kas tieši proporcionāla RF signāla jaudai dBm vienībās.

ADL5500\cite{adl5500} ir RMS jaudas detektors, kas darbojas no 100 MHz līdz 6 GHz. Nodrošina teicamu temperatūras stabilitāti ar gandrīz 0 dB mērījumu kļūdu visā temperatūras diapazonā. Nav plašs mērīšanas diapazons no -25 dB līdz 10 dB.

Tika izvēlēts ADL5906 jaudas detektors, jo piedāvā visplašāko mērīšanas diapozonu no -65 dBm līdz +8 dBm ar temperatūras kompensēšanas shēmu. Frekvenču diapozons atbilst X-joslai, kā arī paredzēts ar modulētiem signāliem ar lielu amplitūdas pīķa-vidējās vērtības attiecību un joslas platumu.