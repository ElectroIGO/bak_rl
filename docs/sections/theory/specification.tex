\section{Vadības bloka specifikācija}
Tehniskā specifikācija tika izvirzīta no VSRC puses, kas tika definēta no izstrādātā HPA. HPA pēdējās pakāpes jaudas pastiprinātāja darba punkta iestatīšanas parametri, sagaidāmā tīkla vadība un jaudas detektora mērīšanas diapazons.
\begin{itemize}
    \item Darba punkta iestatīšanas procedūra (\textit{ang.} Bias-Up procedure)
    \begin{itemize}
        \item Iestatīt jaudas pastiprinātājiem strāvas ierobežojumu uz 18 A, strāvas ierobežojumu uz aizvaru 200 mA.
        \item Iestatīt aizvara spriegumu − 5.0 V.
        \item Pievadīt 24 V spriegumu HPA no elektrobarošanas avota.
        \item Pieskaņot aizvara spriegumu, līdz tiek sasniegta 3 A noteces strāvu.
        \item Pievadīt RF signālu.
    \end{itemize}
    \item Darba punkta atiestatīšana procedūra (\textit{ang.} Bias-Down procedure)
    \begin{itemize}
        \item Samazināt aizvara spriegumu līdz -5 V.
        \item Jaizmēra caurplūstošā strāva caur jaudas pastiprinātāju, jābūt ~ 0 mA.
        \item Jāatvieno barošanas avotu no jaudas pastiprinātāja.
        \item Jāatvieno RF signālu.
    \end{itemize}
    \item Vadība caur tīklu
    \begin{itemize}
        \item TCP servers.
        \item Līdz 8 klientiem.
        \item IPv4.
        \item Jaudas pastiprinātāja un barošanas avota ieslēgšana vai izslēgšana.
        \item Parametru monitorēšana:
        \begin{itemize}
            \item Temperatūra. 
            \item Spriegumi HPA barošanas avotam un vadības signālam (aizvaram).
            \item Caurplūstoša strāva cauri HPA.
            \item Sistēmas stāvoklis.
        \end{itemize}
        \item Kļūdu paziņošana. 
    \end{itemize}
    \item Izstarotās un atstarotās jaudas noteikšana ar RMS RF jaudas detektoru mērīšanas diapazonā no -50 dBm līdz 0 dBm.
\end{itemize}

