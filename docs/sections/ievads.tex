\chapter*{Ievads}
Ventspils Starptautiskā Radioastronomijas Centra (\textit{VSRC}) rīcībā ir teleskopi RT-16 un RT-32, kas vēsturiski tika izmantoti rietumu pasaules spiegošanā, bet pašlaik ir zinātnes instruments kosmosa izpētē. VSRC zinātniskajā institūtā pašlaik notiek darbs pie radioteleskopu komercializācijas, izmantojot antenas kā bāzes stacijas satelītu un kosmosa izpētes misijām, t.sk. uz Mēness. Šim nolūkam viens no VSRC virzieniem ir izveidot S/X diapazona raiduztvērēju RT-16 radioteleskopam, kur daļa no tā ir X diapazona (no 7.145 GHz līdz 7.235 GHz) raidītājs, ko izstrādā šī darba vadītājs Mārcis Bleiders. Bakalaura darba ietvaros ir jāizstrādā raidītāja vadības bloka risinājums, kas paredz sistēmas ieslēgšanu un izslēgšanu X joslas 100 W gallija nitrīda (\textit{GaN}) jaudas pastiprinātāja (\textit{HPA}). Vadības bloks arī ir nepieciešams, lai iestatītu un atiestatītu secīgi darba punktu, pasargātu augstas jaudas pastiprinātāju no pārkaršanas, pārsprieguma, pārstrāvas vai nesalāgotas slodzes pretestības radītiem riskiem, kas var radīt neatgriezeniskus iekārtas bojājumus. 
Bakalaura darba mērķa sasniegšanai tika izvirzīti šādi uzdevumi:
\begin{itemize}
    \item Iegūt specifikāciju jaudas pastiprinātāja vadībai un monitorēšanai no VSRC.
    \item Izpētīt esošās jaudas pastiprinātāju vadības sistēmas.
    \item Piemeklēt piemērotu risinājumu dotajai specifikācijai.
    \item Izveidot testa stendu.
    \item Izstrādāt funkcionējošu maketu, kuru var vadīt caur ethernet tīklu.
    \item Izstrādāt iespiedpalti, piemērotu VSRC vajadzībām.
    \item Integrēt to eksistējošā korpusā.
\end{itemize}
Turpmāk darbs tiek sadalīts vairākās daļās: teorijā, izstrādē un testēšanā. Teorijā tiek padziļinātāk apskatīts idejas koncepts, apskatītie risinājumi un izvēlētā risinājuma teorētiskais pamatojums. Izstrādē tiek detalizēti izklāstīts iekārtas veidošanas process no izstrādes plates līdz paštaisītam risinājumam ar nepieciešamo funkcionalitāti. Testēšanā tiek pārbaudīta iekārtas atbilstība norādītajai specifikācijai.

