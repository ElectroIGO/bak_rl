\newcommand{\abbreviation}[3]{%
    \makebox[4em][l]{#1} --  #2 (\textit{#3})\par
}

\chapter*{Saīsinājumi un to klasifikācija}
\begin{hangingpar}{4em}
\abbreviation{SSC}{Zviedru kosmosa korporācija}{Swedish Space Corporation}
\abbreviation{SPI}{Seriālā perifērā saskarne}{Serial Peripheral Interface}
\abbreviation{VSRC}{Ventspils Starptautiskais Radioastronomijas Centrs}{Ventspils International Radio Astronomy Centre}
\abbreviation{RF}{Radiofrekvence}{Radio frequency}
\abbreviation{GaN}{Gallija nitrīds}{Gallium Nitride}
\abbreviation{DSN}{Dziļā kosmosa tīkls}{Deep Space Network}
\abbreviation{SFCG}{Kosmosa frekvenču koordinēšanas grupa}{Space Frequency Coordination Group}
\abbreviation{ITU}{Starptautiskā Telekomunikāciju savienība}{International Telecommunication Union}
\abbreviation{ADC}{Analogciparu pārveidotājs}{Analog-to-Digital Converter}
\abbreviation{DAC}{Ciparanalogu pārveidotājs}{Digital-to-Analog Converter}
\abbreviation{LNA}{Maztrokšnojošs pastiprinātājs}{Low Noise Amplifier}
\abbreviation{HPA}{Augstas jaudas pastiprinātājs}{High Power Amplifier}
\abbreviation{RX}{Uztveršana}{Reception}
\abbreviation{TX}{Raidīšana}{Transmission}
\abbreviation{ICT}{Informācijas un komunikācijas tehnoloģijas}{Information and Communication Technology}
\abbreviation{CCSDS}{Kosmosa datu sistēmu konsultatīvā komiteja }{Consultative Committee for Space Data Systems}
\abbreviation{NTIA}{Valsts telekomunikāciju un informācijas pārvalde}{National Telecommunications and Information Administration}
\abbreviation{RMS}{Vidējā kvadrātiskā vērtība}{Root mean square}
\abbreviation{IC}{Integrālā shēma}{Integral Circuit}


\end{hangingpar}