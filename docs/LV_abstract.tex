\chapter*{ANOTĀCIJA}
\begin{flushleft}
\textbf{Darba nosaukums:} X joslas raidīšanas sistēmas vadības bloka prototipa izstrāde.\\
\textbf{Darba autors:} Rodrigo Laurinovičs\\
\textbf{Darba vadītājs:} Mg. Sc. Ing. Mārcis Bleiders\\
\textbf{Darba apjoms:} 38.lpp, 5 tabulas, 51 attēli, 31 bibliogrāfiskās norādes, 0 pielikumi\\
\textbf{Atslēgas vārdi:} X-joslas raidītāja darba punkta iestatīšana, RMS jaudas detektors, TCP/IP, satelīta komunikācija\\

Bakalaura darbā izstrādāts funkcionējošs darba punkta un lieljaudas pastiprinātāja monitorēšanas prototips, kas ir vadāms caur tīklu un pielietojams raidītāja vadīšanai satelīta komunikācijās X diapazonā.\\

Darbā aprakstītas RT-16 X diapazona sistēma, dziļā kosmosa komunikāciju sadalījums, jaudas pastiprinātāju esošās vadības sistēmas, jaudas detektori. Definēti sasniedzamie parametri un funkcionalitāte, kas jāpielāgo raidītāju efektīvai izmantošanai Irbenes esošajā radioteleskopu sistēmā.\\

Prototipēšanas atvieglošanas nolūkos katra apakšsistēma izstrādāta kā neatkarīgs modulis, izmantojot mūsdienīgas, viegli pieejamas elektronikas komponentes un materiālus. Izstrādātās moduļu principiālās shēmas izstrādē tika izmantoti datorizēti izstrādes rīki. Izgatavotām apakšsistēmām veikti to raksturīgāko elektrisko parametru mērījumi.\\

Izveidotais prototips notestēts raidīšanas sistēmā, veikti tās veiktspēju raksturojošo parametru mērījumi, pārbaudīta sasniegtā funkcionalitāte un izdarīti secinājumi.
\end{flushleft}

